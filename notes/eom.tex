\documentclass[12pt]{amsart}
\usepackage{amsmath,amssymb,url}
\usepackage{geometry} % see geometry.pdf on how to lay out the page. There's lots.
\geometry{a4paper} % or letter or a5paper or ... etc
% \geometry{landscape} % rotated page geometry

%  POSSIBLY USEFULE PACKAGES
%\usepackage{graphicx}
%\usepackage{tensor}
\usepackage{todonotes}
\usepackage{hyperref}

%  NEW COMMANDS
\newcommand{\pder}[2]{\ensuremath{\frac{ \partial #1}{\partial #2}}}
\newcommand{\ppder}[3]{\ensuremath{\frac{\partial^2 #1}{\partial
      #2 \partial #3} } }
\newcommand{\R}{\ensuremath{\mathbb{R}}}

%  NEW THEOREM ENVIRONMENTS
\newtheorem{thm}{Theorem}[section]
\newtheorem{prop}[thm]{Proposition}
\newtheorem{cor}[thm]{Corollary}
\newtheorem{lem}[thm]{Lemma}
\newtheorem{defn}[thm]{Definition}

%  NEW REMARKS
\theoremstyle{remark}
\newtheorem{rmk}[thm]{Remark}


%%%%%%%%%%%%%%% Holm Macros
\def\MM#1{\boldsymbol{#1}}
\newcommand{\pp}[2]{\frac{\partial #1}{\partial #2}} 
\newcommand{\dede}[2]{\frac{\delta #1}{\delta #2}}
\newcommand{\dd}[2]{\frac{\diff #1}{\diff #2}} 
\newcommand{\sothree}{\mathfrak{so}(3)}
% \contract is a differential geometry contraction sign _|
\def\contract{\makebox[1.2em][c]{\mbox{\rule{.6em}
{.01truein}\rule{.01truein}{.6em}}}}
%%%%%%%%%%%%%%%%%%%%%%%%%%%%%%%%%%%%%%%%%%%

%  MATH OPERATORS
\DeclareMathOperator{\SDiff}{SDiff}
\DeclareMathOperator{\GL}{GL}
\DeclareMathOperator{\SO}{SO}
\DeclareMathOperator{\ad}{ad}
\DeclareMathOperator{\Ad}{Ad}
\DeclareMathOperator{\Jet}{Jet}
\DeclareMathOperator{\Orb}{Orb}

%  TITLE, AUTHOR, DATE
\title{Equations of motion for Jet vortex methods}
\author{Darryl D. Holm \& Henry O. Jacobs}
\date{\today}


\begin{document}

\maketitle
We derive the equations of motion for jet-vortices.

\subsection{0th order}
Consider the ansatz for the vorticity
\begin{align}
  \omega(t) = \sum_{i} \Gamma_i(t) \delta_{z_i} \label{eq:ansatz 0}
\end{align}
for constants $\Gamma_i(t) \in \R$ for $i$ in some finite set $S$
and where the points $z_i \in \R^2$ are distinct.
We then find
\begin{align}
  \partial_t \omega = \sum_{i} \frac{d \Gamma_i}{dt} \delta_{z_i} 
  - \Gamma_i \frac{dx_i}{dt} \partial_x \delta_{z_i}
  - \Gamma_i \frac{dy_i}{dt} \partial_y \delta_{z_i}.
  \label{eq:time derivative}
\end{align}
and
\begin{align*}
  u \partial_x \omega = \sum_{i} u \Gamma_i \partial_x \delta_{z_i}.
\end{align*}
Here $u \partial_x \delta$ should be interpreted
in terms of the basic theory of distrbutions.
Invoking \eqref{eq:func times partial delta} (see Appendix
\ref{sec:distributions}) we find
\begin{align}
  u \partial_x \omega = \sum_{i} \Gamma_i u(z_i) \partial_x \delta_{z_i} - \partial_x u(z_i) \delta_{z_i}. \label{eq:u term}
\end{align}
Similarly
\begin{align}
    v \partial_y \omega = \sum_{i} \Gamma_i v(z_i) \partial_y \delta_{z_i} - \partial_y v(z_i) \delta_{z_i}. \label{eq:v term}
\end{align}
Substituion of \eqref{eq:time derivative},\eqref{eq:u term},
and \eqref{eq:v term} into \eqref{eq:vorticity} yields
\begin{align*}
\frac{d \Gamma_i}{dt} \delta_{z_i} 
  - \Gamma_i \frac{dx_i}{dt} \partial_x \delta_{z_i}
  - \Gamma_i \frac{dy_i}{dt} \partial_y \delta_{z_i}
  + \Gamma_i u(z_i) \partial_x \delta_{z_i} - \partial_x u(z_i) \delta_{z_i}\\
  \quad + \Gamma_i v(z_i) \partial_y \delta_{z_i} - \partial_y v(z_i) \delta_{z_i} =  0.
\end{align*}
We observe this is a linear combination of the distributions $\delta_{z_i},\partial_x\delta_{z_i}$ and $\partial_y \delta_{z_i}$.
The coefficients of each must vanish independently.
The vanishing of the coefficient of $\delta_{z_i}$ yields
\begin{align*}
  \frac{d\Gamma_i}{dt} = \partial_x u(z_i) + \partial_y v(z_i).
\end{align*}
As the flow is divergence free we find
\begin{align*}
  \frac{d\Gamma_i}{dt} = 0.
\end{align*}
The vanishing of the coefficients $\partial_x \delta_{z_i}$ and $\partial_y \delta_{z_i}$ yields
\begin{align}
  \frac{dx_i}{dt} = u(z_i) \quad , \quad \frac{dy_i}{dt} = v(z_i).
  \label{eq:point motion}
\end{align}
Thus we have found equations of motion for $x_i(t),y_i(t)$ and $\Gamma_i(t)$
whose solutions make \eqref{eq:ansatz 0} into a solution of \eqref{eq:vorticity}.

\subsection{1st order}
Consider the first order jet-vortex ansatz for the vorticity
\begin{align}
  \omega(t) = \sum_{i} \Gamma_i(t) \delta_{z_i} 
  + \Gamma_i^x \partial_x \delta_{z_i}
  + \Gamma_i^y \partial_y \delta_{z_i}\label{eq:ansatz 1}
\end{align}
for constants $\Gamma_i(t),\Gamma_i^x(t),\Gamma_i^y(t) \in \R$ for $i = 1,\dots,N$.
We then find
\begin{align*}
  \partial_t \omega = \sum_{i} & \frac{d \Gamma_i}{dt} \delta_{z_i} 
  - \Gamma_i \frac{dx_i}{dt} \partial_x \delta_{z_i}
  - \Gamma_i \frac{dy_i}{dt} \partial_y \delta_{z_i} \\
  &\frac{d \Gamma_i^x}{dt} \partial_x\delta_{z_i} 
  - \Gamma_i^x \frac{dx_i}{dt} \partial_x^2 \delta_{z_i}
  - \Gamma_i^x \frac{dy_i}{dt} \partial_{xy} \delta_{z_i} \\
  &\frac{d \Gamma_i^y}{dt} \delta_{z_i} 
  - \Gamma_i^y \frac{dx_i}{dt} \partial_{xy} \delta_{z_i}
  - \Gamma_i^y \frac{dy_i}{dt} \partial_y^2 \delta_{z_i}
\end{align*}
and
\begin{align*}
  u \partial_x \omega = \sum_{i} u \Gamma_i \partial_x \delta_{z_i}
  + u \Gamma_i^x \partial_x^2 \delta_{z_i} 
  + u \Gamma_i^y \partial_{xy} \delta_{z_i}.
\end{align*}
Invoking \eqref{eq:func times partial delta} (see Appendix
\ref{sec:distributions}) we can revise the previous equation to
obtain
\begin{align*}
  u \partial_x \omega &= \sum_{i} \Gamma_i ( u(z_i) \partial_x \delta_{z_i} - \partial_x u(z_i) \delta_{z_i} ) \\
   &\Gamma_i^x (u(z_i) \partial_x^2 \delta_{z_i} 
   - 2\partial_x u(z_i) \partial_x \delta_{z_i}
   + \partial_x^2u(z_i) \delta_{z_i} ) \\
   &\Gamma_i^y ( u(z_i) \partial_{xy} \delta_{z_i} 
   - \partial_x u(z_i) \partial_y \delta_{z_i} 
   - \partial_y u(z_i) \partial_x \delta_{z_i} 
   + \partial_{xy}u(z_i) \delta_{z_i}
\end{align*}
Similarly
\begin{align*}
    v \partial_y \omega &= \sum_{i} \Gamma_i ( v(z_i) \partial_y \delta_{z_i} - \partial_y v(z_i) \delta_{z_i} ) \\
   &\quad \Gamma_i^x (v(z_i) \partial_{xy} \delta_{z_i} 
   - \partial_x v(z_i) \partial_x \delta_{z_i}
   - \partial_y v(z_i) \partial_x \delta_{z_i}
   + \partial_{xy} v(z_i) \delta_{z_i} ) \\
   &\quad \Gamma_i^y ( v(z_i) \partial_{y}^2 \delta_{z_i} 
   - 2\partial_y v(z_i) \partial_y \delta_{z_i} 
   + \partial_{y}^2v(z_i) \delta_{z_i})
\end{align*}
Substituion of these expressions into \eqref{eq:vorticity} yields
the vanishing of
a linear combination of the distributions
$\delta_{z_i},\partial_x\delta_{z_i},\partial_y \delta_{z_i}, \partial_x^2 \delta_{z_i},
 \partial_{xy} \delta_{z_i}$, and $\partial_y^2 \delta_{z_i}$.
As each of these distributions is linearly independent
of the others (assuming the $z_i$'s are distinct)
we find that the coefficients must vanish independently.
The vanishing of the coefficient of $\delta_{z_i}$ yields
the equation
\begin{align*}
  \frac{d\Gamma_i}{dt} = \partial_x u(z_i) + \partial_y v(z_i)
  + \partial_x^2 u(z_i) + \partial_{xy} v(z_i)
  + \partial_{xy} u(z_i) + \partial_y^2 v(z_i)
\end{align*}
Again, as the flow is divergence free we find
\begin{align*}
  \frac{d\Gamma_i}{dt} = 0.
\end{align*}
The vanishing of the coefficients of $\partial_x^2 \delta_{z_i}$ yields
\begin{align*}
  \Gamma_i^x ( u(z_i) - \frac{dx_i}{dt} ) = 0.
\end{align*}
Similarly, by looking at the coefficient of $\partial_y^2 \delta_{z_i}$
we find
\begin{align*}
  \Gamma_i^y ( v(z_i) - \frac{dy_i}{dt} ) = 0.
\end{align*}
Assuming $\Gamma_i^x$ and $\Gamma_i^y$ are non-zero we
obtain \eqref{eq:point motion} again.\footnote{
  After we obtain dynamics under this assumption
  we may drop the assumption by continous extension.
  The resulting extension yields the $0$th order
  jet-vortex dynamics.
}

Finally, the vanishing of the coefficients $\partial_x \delta_{z_i}$ and $\partial_y \delta_{z_i}$ yields
\begin{align*}
  \frac{d\Gamma^x_i}{dt} = \Gamma_i^x \partial_x u(z_i) + \Gamma_i^y \partial_y u(z_i) \\
  \frac{d\Gamma^y_i}{dt} = \Gamma_i^x \partial_x v(z_i) + \Gamma_i^y \partial_y v(z_i)
\end{align*}
Thus we have found equations of motion for $x_i(t),y_i(t)$ and the
$\Gamma_i(t)$'s.
The solutions of this finite-dimensional ODE
will make the first order jet-vortex ansatz, \eqref{eq:ansatz 1},
into a solution of \eqref{eq:vorticity}.

\subsection{$N$th order}
Consider the $N$th order jet-vortex ansatz for the vorticity
\begin{align}
  \omega(t) = \sum_{i \in S} \sum_{m+n \leq N} \Gamma^{mn}_i(t) \partial_x^m \partial_y^n \delta_{z_i} \label{eq:ansatz N}
\end{align}
for constants $\Gamma^{mn}_i(t),\Gamma_i^x(t),\Gamma_i^y(t) \in \R$ for $i \in S$.
We then find
\begin{align*}
  \partial_t \omega =	
  \sum_{
  	\substack{
		i \in S \\
		m+n \leq N}}
  	\frac{d \Gamma_i^{mn}}{dt} \partial_x^m \partial_y^n \delta_{z_i} - \Gamma_i^{mn} \frac{dx_i}{dt} \partial_{x}^{m+1} \partial_y^{n} \delta_{z_i}
	- \Gamma_i^{mn} \frac{dy_i}{dt} \partial_{x}^{m} \partial_y^{n+1} \delta_{z_i}.
\end{align*}
and
\begin{align*}
  u \partial_x \omega = 
  \sum_{
  	\substack{
		i \in S \\
		m+n \leq N}}
   u \Gamma_i ^{mn}\partial_x^{m+1} \partial_y^n \delta_{z_i}.
\end{align*}
Invoking \eqref{eq:func times partial delta} (see Appendix
\ref{sec:distributions}) we can revise the previous equation to
obtain
\begin{align*}
  u \partial_x \omega &=
  \sum_{
  	\substack{
		i \in S \\
		m+n \leq N}}
	\Gamma_i^{mn} (-1)^{m+n+1} \sum_{\ell,k} (-1)^{\ell + k} \binom{m+1}{\ell} \binom{n}{k} \partial_x^{\ell} \partial_y^k u(z_i) \partial_x^{m+1-\ell} \partial_y^{n-k} \delta_{z_i}.
\end{align*}
Similarly
\begin{align*}
  v \partial_y \omega &=
  \sum_{
  	\substack{
		i \in S \\
		m+n \leq N}}
	\Gamma_i^{mn} (-1)^{m+n+1} \sum_{\ell,k} (-1)^{\ell + k} \binom{m}{\ell} \binom{n+1}{k} \partial_x^{\ell} \partial_y^k v(z_i) \partial_x^{m-\ell} \partial_y^{n+1-k} \delta_{z_i}.
\end{align*}
Substitution of these expressions into \eqref{eq:vorticity} yields
the vanishing of a linear combination of the distributions
$\partial_x^m \partial_y^n \delta_{z_i}$
for $m+n \leq N+1$.
As each of these distributions is linearly independent
of the others (assuming the $z_i$'s are distinct)
we find that the coefficients must vanish independently.
Assuming sufficiently many of the $\Gamma_i^{m,n}$'s are non-zero 
for $m+n = N$ we recover \eqref{eq:point motion}.\footnote{
  As before, after we obtain dynamics for the $\Gamma$'s under this assumption
  we may drop the assumption by continuous extension.
  The resulting extension yields the $0$th order
  jet-vortex dynamics.
}
The vanishing of the coefficient of $\delta_{z_i}$ yields
\begin{align*}
	\frac{d\Gamma^{0,0}_i}{dt} + \sum_{0 < m+n \leq N} \Gamma^{m,n} ( \partial_x^{m+1}\partial_y^n u(z_i) + \partial_x^{m}\partial_y^{n+1} v(z_i) ) = 0.
\end{align*}
As $u$ is divergence free this equation reduces to
\begin{align*}
	\frac{d \Gamma^{0,0} }{dt} = 0
\end{align*}

For $\ell+k \leq N$ the vanishing of the coefficient of $\partial_{\ell,k} \delta_{z_i}$ yields
\begin{align*}
  \dot{\Gamma}_i^{\ell k} = (-1)^{\ell + k}
  \sum_{
    \substack{
      m > \ell \\
      n > k \\
      n+m \leq p}
    }\Gamma_i^{mn} \Big[  \binom{n}{k} \binom{m}{\ell-1} \partial_{m-\ell+1,n-k} u^x(z_i) \\
   \binom{n}{k-1} \binom{m}{\ell} \partial_{m-\ell,n-k+1} u^y(z_i)  \Big]
\end{align*}
Again, we have found equations of motion for $x_i(t),y_i(t)$ and the
$\Gamma_i(t)$'s.
The solutions of this finite-dimensional ODE
will make the first order jet-vortex ansatz, \eqref{eq:ansatz 1},
into a solution of \eqref{eq:vorticity}.

\begin{rmk}
  We will find that the following quantities are conserved:
  
  \begin{tabular}{|c|c|}
  	\hline
	  Linear momentum & $J_{lin} = \sum_i ( \Gamma_i^{0,1} - \Gamma^{0,0}_i y_i , \Gamma^{0,0}_i x_i -\Gamma^{1,0}_i )$ \\
	  \hline
	  Angular momentum & $J_{ang} = \sum_i \frac{\Gamma^{0,0}_i}{2} (x_i^2 + y_i^2) - \Gamma_i^{1,0} x_i - \Gamma_i^{0,1} y_i + \Gamma_i^{2,0} + \Gamma_i^{0,2}$ \\
	  \hline
	  Energy & $E = \sum_{m,n,\ell,k,i} (-1)^{m+n+\ell +k} \Gamma^{mn}_i \Gamma_j^{\ell k} \partial_{m+\ell}^x \partial_{n+k}^y G(z_i-z_j)$ \\
	  \hline
  \end{tabular}

  The first two are momenta derived by observing the rotational and
  translational symmetry of the fluid and applying Noether's theorem (see Appendix \ref{sec:symmetries} for derivation).
\end{rmk}

\begin{rmk}
  Moments ... \todo[inline]{Still need to write the moments section.}
\end{rmk}


\appendix

\section{Distributions}
\label{sec:distributions}
The vorticity, $\omega$, should be viewed as a distribution
and the term ``$\partial_x \omega$'' should be viewed
as a distributional derivative.
When $\omega \in D(\R^2)$ is a smooth distribution there is little harm in naively interpreting $\omega$ as a smooth function on $\R^2$.
However, when $\omega$ is not smooth (e.g. a Dirac delta distribution),
then one really needs to invoke the mathematics of distributions
as distinct from that of real valued functions.
Therefore, we have included this appendix to remind the reader of the basic theory of distributions.
The main reference for this section is \cite{Hormander2003}.\todo{We should refer to specific formulas from here.}

The space of distributions $D(\R^2)$ is the dual-vector space to the space of smooth functions $C^\infty(\R^2)$.
Therefore a distribution is defined by how it maps functions to real numbers.
Given a distribution $\omega \in D(\R^2)$ and a function $f \in C^\infty(\R^2)$,
we use the notation $\langle \omega , f \rangle$ to denote the real number $\omega(f)$.

Given a distribution $\omega \in D(\R^2)$.
We can define the distributional derivative of $\omega$ in the $i$th coordinate direction
as the distribution $\partial_i \omega$ defined by
\begin{align*}
	\langle \partial_i \omega , f \rangle = - \langle \omega , \partial_i f \rangle.
\end{align*}
For example, the Dirac-delta distribution, $\delta_0$, is defined as the unique distribution such that
\begin{align*}
	\langle \delta_0 , f \rangle = f(0) \quad , \quad \forall f \in C^\infty(\R^2).
\end{align*}
The distributional derivative of $\partial_i \delta_0$ is given by
\begin{align*}
	\langle \partial_i \delta_0 , f \rangle = -\partial_i f(0) \quad , \quad \forall f \in C^\infty(\R^2).
\end{align*}
Given a distribution $\omega \in D(\R^2)$ and a function $g \in C^\infty(\R^2)$ we can define the
distribution $g \omega$ given by
\begin{align*}
	\langle g \omega , f \rangle = \langle \omega , gf \rangle \quad , \quad \forall f \in C^\infty(\R^2).
\end{align*}
For example, we find that $g \delta _0 = g(0) \delta_0$.
A slightly more involved example is given by the computation of $g \partial_i \delta_0$.
We find
\begin{align*}
	\langle g \partial_i \delta_0 ,  f \rangle = \langle \partial_i \delta_0 , gf \rangle =  - g(0) \partial_i f(0) - \partial_ig(0) f(0).
\end{align*}
Therefore
\begin{align*}
	g \partial_i \delta_0 = g(0) \partial_i \delta_0 - \partial_i g(0) \delta_0.
\end{align*}
On the left hand side, note that $g(0)$ and $\partial_i g(0)$ are merely real numbers,
which are multiplying the distributions $\partial_i \delta_0$ and $\delta_0$.
More generally, we find
\begin{align*}
	\langle g \partial_x^m \partial_y^n \delta_0 , f \rangle &= (-1)^{m+n} \partial_x^m \partial_y^n (fg)(0) \\
		&= (-1)^{m+n} \sum_{\ell,k=0}^{m,n} \binom{m}{\ell} \binom{n}{k}
		\left(\partial_{x}^{\ell} \partial_y^k f(0) \right) 
		\left(\partial_{x}^{m-\ell} \partial_y^{n-k} g(0) \right) 
\end{align*}
which means
\begin{align}
	g \partial_x^m \partial_y^n \delta_0 =
		(-1)^{m+n} \sum_{\ell,k=0}^{m,n} (-1)^{\ell + k}
		\binom{m}{\ell} \binom{n}{k}
		\left(\partial_{x}^{m-\ell} \partial_y^{n-k} g(0) \right) 
		\partial_{x}^{\ell} \partial_y^k \delta_0.
                \label{eq:func times partial delta}
\end{align}
\end{thebibliography}

\bibliographystyle{amsalpha}
\bibliography{hoj_2015}
\end{document}
