\documentclass[12pt]{amsart}
\usepackage{amsmath,amssymb}
\usepackage{geometry} % see geometry.pdf on how to lay out the page. There's lots.
\geometry{a4paper} % or letter or a5paper or ... etc
% \geometry{landscape} % rotated page geometry

%  POSSIBLY USEFULE PACKAGES
%\usepackage{graphicx}
%\usepackage{tensor}
%\usepackage{todonotes}

%  NEW COMMANDS
\newcommand{\pder}[2]{\ensuremath{\frac{ \partial #1}{\partial #2}}}
\newcommand{\ppder}[3]{\ensuremath{\frac{\partial^2 #1}{\partial
      #2 \partial #3} } }

%  NEW THEOREM ENVIRONMENTS
\newtheorem{thm}{Theorem}[section]
\newtheorem{prop}[thm]{Proposition}
\newtheorem{cor}[thm]{Corollary}
\newtheorem{lem}[thm]{Lemma}
\newtheorem{defn}[thm]{Definition}


%  MATH OPERATORS
\DeclareMathOperator{\Diff}{Diff}
\DeclareMathOperator{\GL}{GL}
\DeclareMathOperator{\SO}{SO}
\DeclareMathOperator{\ad}{ad}
\DeclareMathOperator{\Ei}{Ei}

%  TITLE, AUTHOR, DATE
\title{Rescaling notes}
\author{Henry O. Jacobs}
\date{\today}


\begin{document}

\maketitle

In these notes we record the change of variables needed to compute
the flow given an initial stream-function $\psi_0(x,y)$.
Let us coordinatize space where $\delta \ll 1$ by $(x,y)$
and the space where $\delta = 1$ by $(X,Y)$.
We see that $X = \delta^{-1} x$ and $Y = \delta^{-1}y$.

The Green's kernel is
\begin{align*}
  G_\delta(x,y) &= \frac{1}{4 \pi}( \Ei(-r^2 / \delta^2) - 2\log(r)) \\
  &= \frac{1}{4\pi} ( \Ei(-R^2)-2\log(\delta R)) \\
  &= \frac{1}{4\pi} ( \Ei(R^2) -2 \log(\delta) - 2 \log(R) ) \\
  &= G_1(X,Y) - \frac{1}{2\pi} \log(\delta) \\
\end{align*}
This allows us to write $G_\delta$ and it's derivatives in terms
of $G_1$ which is numerically more stable to calculate.
We find that
\begin{align*}
  \partial_x G_\delta = \partial_x G_1 = \partial_X G_1 \cdot \frac{\partial X}{\partial x} = \delta^{-1} \partial_X G_1.
\end{align*}
More generally
\begin{align*}
  \partial_x^k G_\delta = \delta^{-k} \partial_X^k G_1
\end{align*}


\section{0th order}
Given $\psi(x,y)$ we desire to solve the inverse problem
\begin{align*}
  \psi(z_i) = \sum_j \Gamma_j G_\delta( z_i - z_j).
\end{align*}
Conversion to $Z$ coordinates we find
\begin{align*}
  \psi(\delta Z_i) = \left(\sum_j \Gamma_j G_1( Z_i - Z_j) \right)- \frac{N}{2\pi} \log(\delta).
\end{align*}
Therefore, to obtain $\Gamma_j$ using just $G_1$
it is sufficient to define the function $\Psi(Z) := \psi(\delta Z) + \frac{N}{2\pi} \log(\delta)$
and solve the inverse problem
\begin{align*}
  \Psi(Z_i) = \sum_j \Gamma_j G_1(Z_i - Z_j).
\end{align*}
This produces the estimate of $\Psi$ given by
\begin{align*}
  \tilde{\Psi}(Z) = \sum_j \Gamma_j G_1(Z - Z_j).
\end{align*}


\section{1st order}
In this case we wish to solve the inverse problem
\begin{align*}
  \psi(z_i) &= \sum_{j} \Gamma_j G_\delta(z_i - z_j)
  + \Gamma_j^x \partial_x G_\delta(z_i - z_j)
  + \Gamma_j^y \partial_y G_\delta(z_j - z_j) \\
  \partial_x \psi(z_i) &=  \sum_j \Gamma_j \partial_x G_\delta(z_i - z_j)
  + \Gamma_j^x \partial_x^2 G_\delta(z_i - z_j)
  + \Gamma_j^y \partial_{xy} G_\delta(z_j - z_j) \\
  \partial_y \psi(z_i) &=  \sum_j \Gamma_j \partial_y G_\delta(z_i - z_j)
  + \Gamma_j^x \partial_{xy} G_\delta(z_i - z_j)
  + \Gamma_j^y \partial_{y}^2 G_\delta(z_j - z_j)
\end{align*}
In $Z$ coordinates with $\Psi$ as before we find this is
\begin{align*}
  \Psi(Z_i) &= \sum_{j} \Gamma_j G_1(Z_i - Z_j)
  + \Gamma_j^X \partial_X G_1(Z_i - Z_j)
  + \Gamma_j^Y \partial_Y G_1(Z_j - Z_j) \\
  \partial_X \Psi(Z_i) &=  \sum_j \Gamma_j \partial_X G_1(Z_i - Z_j)
  + \Gamma_j^X \partial_{X}^2 G_1(Z_i - Z_j)
  + \Gamma_j^Y \partial_{XY} G_1(Z_j - Z_j) \\
  \partial_Y \Psi(Z_i) &=  \sum_j \Gamma_j \partial_Y G_1(Z_i - Z_j)
  + \Gamma_j^X \partial_{XY} G_1(Z_i - Z_j)
  + \Gamma_j^Y \partial_{Y}^2 G_1(Z_j - Z_j)
\end{align*}
Where $\Gamma_j^X = \delta^{-1} \Gamma_j^x$.

This produces the estimate
\begin{align*}
  \tilde{\Psi}(Z) = \sum_j \Gamma_j G_1(Z-Z_j) + \Gamma_j^X \partial_X G_1(Z-Z_j).
\end{align*}


\subsection{Second order}
Taking a wild guess, I'd imagine we still use $\Psi$ 
and define $\Gamma_i^{XX} = \delta^{-2} \Gamma_{i}^{xx}$, etc.

\section{The big picture}
We desire to solve for $\tilde{\psi}(z)$
using evaluations of $G_1$ rather than $G_\delta$.
To do this solve for $\tilde{\Psi}(Z)$ first.
\begin{align*}
  \tilde{\Psi}(Z) = \sum_j \Gamma_j G_1(Z-Z_j) 
  + \Gamma_j^X \partial_X G_1(Z-Z_j) + \dots
\end{align*}
Then
\begin{align*}
  \tilde{\psi}(z) = \tilde{\Psi}( \delta^{-1} Z)  - \frac{N \log(\delta)}{2\pi}.
\end{align*}



\bibliographystyle{amsalpha}
\bibliography{/Users/hoj201/Dropbox/hoj_2014.bib}
\end{document}
