\documentclass[12pt]{amsart}
\usepackage{amsmath,amssymb}
\usepackage{geometry} % see geometry.pdf on how to lay out the page. There's lots.
\geometry{a4paper} % or letter or a5paper or ... etc
% \geometry{landscape} % rotated page geometry

%  POSSIBLY USEFULE PACKAGES
%\usepackage{graphicx}
%\usepackage{tensor}
%\usepackage{todonotes}

%  NEW COMMANDS
\newcommand{\pder}[2]{\ensuremath{\frac{ \partial #1}{\partial #2}}}
\newcommand{\ppder}[3]{\ensuremath{\frac{\partial^2 #1}{\partial
      #2 \partial #3} } }

%  NEW THEOREM ENVIRONMENTS
\newtheorem{thm}{Theorem}[section]
\newtheorem{prop}[thm]{Proposition}
\newtheorem{cor}[thm]{Corollary}
\newtheorem{lem}[thm]{Lemma}
\newtheorem{defn}[thm]{Definition}


%  MATH OPERATORS
\DeclareMathOperator{\Diff}{Diff}
\DeclareMathOperator{\GL}{GL}
\DeclareMathOperator{\SO}{SO}
\DeclareMathOperator{\ad}{ad}
\DeclareMathOperator{\Ad}{Ad}

%  TITLE, AUTHOR, DATE
\title{My vision of jet vortices}
\author{Henry O. Jacobs}
\date{\today}


\begin{document}

\maketitle

The vortex blob method is one of very few well established meshless methods in CFD.
Moreover, the convergence of the vortex blob method is spectral, and the (thanks to the fast multipole method), the computational complexity 
at each time-step is $O(n \log(n))$.
However, there is room for improvement.
When vortices collide or get close, the speed of the vorticies can become unreasonably large for numerical implementation.

The jet-vortex algorithm yields the following benefits:
\begin{itemize}
  \item Faster convergence (observed numerically)
  \item Resolves collisions (through vortex mergers)
  \item Retains the benefits of the vortex method ($O(n \log(n))$ complexity, spectral convergence, meshless)
\end{itemize}

The foundations of the method are just as secure as that of the theory of vortex-blobs studied in \cite{MarsdenWeinstein1983,GayBalmazVizman2012}.  We are able to derive the space of jet-vortices as a (finite-dimensional) coadjoint orbit of $\mathfrak{X}(\mathbb{R}^2)^*$.
Conserved quantities such as angular and linear momentum are still conserved,
and the conservation of circulation is still present.\footnote{This is because the jet-vortex solutions solve the PDE
\begin{align*}
  \partial_t \omega + \pounds_{\nabla \perp \psi}[\omega]
  \quad,\quad \psi = K^* \omega
\end{align*}
so that $\omega(t) = (\Phi_t)_*\omega(0)$ where $\Phi_t$ is the flow of the vector-field $u = \nable^\perp \psi$.
Which means $(\Phi_t)^* \omega(t)$ is constant.
}
In summary, jet-vortex methods have all the benefits of vortex blob methods,
and more.

\bibliographystyle{amsalpha}
\bibliography{/Users/hoj201/Dropbox/hoj_2015.bib}
\end{document}
