\documentclass[12pt]{amsart}
\usepackage{amsmath,amssymb}
\usepackage{geometry} % see geometry.pdf on how to lay out the page. There's lots.
\geometry{a4paper} % or letter or a5paper or ... etc
% \geometry{landscape} % rotated page geometry

%  POSSIBLY USEFULE PACKAGES
%\usepackage{graphicx}
%\usepackage{tensor}
\usepackage{todonotes}

%  NEW COMMANDS
\newcommand{\pder}[2]{\ensuremath{\frac{ \partial #1}{\partial #2}}}
\newcommand{\ppder}[3]{\ensuremath{\frac{\partial^2 #1}{\partial
      #2 \partial #3} } }
\newcommand{\R}{\ensuremath{\mathbb{R}}}

%  NEW THEOREM ENVIRONMENTS
\newtheorem{thm}{Theorem}[section]
\newtheorem{prop}[thm]{Proposition}
\newtheorem{cor}[thm]{Corollary}
\newtheorem{lem}[thm]{Lemma}
\newtheorem{defn}[thm]{Definition}


%  MATH OPERATORS
\DeclareMathOperator{\SDiff}{SDiff}
\DeclareMathOperator{\GL}{GL}
\DeclareMathOperator{\SO}{SO}
\DeclareMathOperator{\ad}{ad}
\DeclareMathOperator{\Ad}{Ad}
\DeclareMathOperator{\Jet}{Jet}
\DeclareMathOperator{\Orb}{Orb}

%  TITLE, AUTHOR, DATE
\title{Jet vortex methods}
\author{Darryl D. Holm, Henry O. Jacobs}
\date{\today}


\begin{document}

\maketitle

\begin{abstract}
  blah blah blah.
\end{abstract}

\section{Introduction}
\label{sec:intro}

\todo[inline]{Darryl writes this}

\subsection{Notation}
The symbol ``$z$'' will be used to denote
an element of $\mathbb{R}^2$
with coordinates $z = (x,y)$.
The Dirac-delta distribution on $\mathbb{R}^2$
will be denoted by $\delta(\cdot)$, and $\delta_z(\cdot) := \delta( \cdot - z)$.
The Laplace operator will be denoted by $\Delta := \partial_x^2 + \partial_y^2$.

\section{Background}

\todo[inline]{Darryl write this.  Should include vortex method literature survey and regularized fluid modeling stuff.}

\section{The vorticity equation for regularized fluids}
\label{sec:vorticity}
The equation of motion we wish to study are the planar ideal fluid equations
\begin{align*}
  \partial_t u + u \cdot \nabla u = - \nabla p \quad, \quad \nabla \cdot u = 0,
\end{align*}
where $u:\R^2 \to \R^2$ is a vector field.
More generally, we may consider a regularized model of ideal fluids given by
\begin{align*}
  \partial_t m + u \cdot \nabla m = - dp \quad,\quad \nabla \cdot u = 0, m = L[u]
\end{align*}
where $L$ is a psuedo-differential operator (e.g. a Helmholtz operator)
which is ``close to the identity'' in some reasonable sense.
For example $L = (1-\alpha \Delta)$ yeilds the Euler-$\alpha$ model.\todo{add citation}
We define the vorticity $\omega = {\rm curl}(m) = \partial_x m^y - \partial_y m^x$.
If we let $\varphi_t : \R^2 \to \R^2$ denote the flow of $u$ at time $t$
and let $\omega_t$ denote the vorticity at time $t$, then
\begin{align*}
  \omega_t (x):= \det[ D\varphi_t^{-1}(x) ] \omega_0(\varphi_t^{-1}(x)).
\end{align*}
Individuals who have transformed a probability distribution should be well aquainted with the above
equation.
If we view $\omega$ as a distribution,
this equation tells us that $\omega$ is advected by the flow of $u$.
It is geometrically insightful to define an operator $(\varphi_t)_*$
associated to $\varphi_t$ and write $\omega_t = (\varphi_t)_* \omega_0$.
This operator is called the \emph{push-forward}, and appears everywhere in differential geometry
when objects (functions, tensors, distributions,etc.) are advected by a diffeomorphism.\todo{cite something}

In any case, by taking a time-derivative of $\omega_t$ we obtain the advection equation for the vorticity
\begin{align*}
  \partial_t \omega + \pounds_u [\omega] = 0
\end{align*}
Where $\pounds_u[\omega]$ is the unique distribution defined by the property
\begin{align*}
  \langle \pounds_u[\omega] , f \rangle = - \langle \omega , u^i \partial_i f \rangle
\end{align*}
This is the vorticity equation, expanded to the case of distributions.


\section{Derivation of the vortex blob method}
\label{sec:vortex_blob}
In this section we derive the equations of motion for the traditional
vortex blob method \todo{cite stuff}.
We can begin by considering the vorticity formulation of Euler's equations for an incompressible inviscid planar fluid
\begin{align}
  &\partial_t \omega + \pounds_u[ \omega] = 0 \label{eq:eom1} \\
  &\omega = (L \circ \Delta) \cdot \psi \label{eq:eom2} \\
  &u = (u^1,u^2) = \nabla^\perp \psi :=  ( \partial_y \psi, - \partial_x \psi)
  \label{eq:eom3}
\end{align}
By viewing the vorticity, $\omega$, is as a distribution,
we may consider the ansatz
\begin{align}
  \omega(z) = \sum_{i=1}^N \Gamma_i(t) \delta(z-z_i(t))
  \label{eq:ansatz}
\end{align} 
with $\Gamma_i(t) \in \mathbb{R}$,
and $z_i(t) \in \mathbb{R}^2$.
Our desire is to obtain an ordinary differential equation for the $\Gamma_i$'s and the $z_i$'s,
whose solutions make the $\omega$ of \eqref{eq:ansatz} a solution of \eqref{eq:eom1} and \eqref{eq:eom2}.
To declutter future formulas,
we will suppress the time-dependence of $z_i$ and $\Gamma_i$
from this point onwards.

In order to find the desired ordinary differential equation for the $\Gamma_i$'s and $z_i$'s
we will seek consistency constraints implied by the equations of motion
\eqref{eq:eom1},\eqref{eq:eom2}. Substition of \eqref{eq:ansatz} into \eqref{eq:eom2} imples
\begin{align*}
  \psi = \sum_{i=1}^N \Gamma_i G_{z_i}
\end{align*}
where $G_{z}(\cdot) := G( \cdot - z)$ and  $G(\cdot)$ is the Green's function of $L \circ \Delta$.
By \eqref{eq:eom3} this implies, $u$, must be
\begin{align}
  u(z) =  \sum_{i=1}^N \begin{pmatrix}
    \Gamma_i \partial_y G_{z_i} \\
    -\Gamma_i \partial_x G_{z_i}
  \end{pmatrix}. \label{eq:u}
\end{align}
Now we will compute the term ``$\pounds_u[\omega]$'' found in \eqref{eq:eom1}.
Given an arbitrary function $f \in C^\infty(\mathbb{R}^2)$ we may invoke the definition of $\pounds_u[\cdot]$ to find
\begin{align*}
  \langle \pounds_u[ \Gamma_i \delta_{z_i} ] , f \rangle &= - \Gamma_i u^x(z_i) \partial_x f(z_i) - \Gamma_i u^y(z_i) \partial_y f(z_i) \\
 &= \langle \Gamma_i u^x(z_i) \partial_x \delta_{z_i} + \Gamma_i u^y(z_i) \partial_y \delta_{z_i} , f \rangle
\end{align*}
In other words
\begin{align}
  \pounds_u[\Gamma_i \delta_{z_i}] =  \Gamma_i u^x(z_i) \partial_x \delta_{z_i} + \Gamma_i u^y(z_i) \partial_y \delta_{z_i}. \label{eq:Lie}
\end{align}
More-over, our assumption regarding the ansatz implies
\begin{align}
  \partial_t \omega = \sum_{i=1}^N \dot{\Gamma}_i \delta_{z_i} - \Gamma_i \dot{x}_i \partial_x \delta_{z_i} - \Gamma_i \dot{y}_i \partial_y \delta_{z_i}. \label{eq:time_derivative}
\end{align}
Finally, substitution of \eqref{eq:Lie} and \eqref{eq:time_derivative} into \eqref{eq:eom1} yields
\begin{align}
  \begin{split}
  & \sum_{i=1}^N \dot{\Gamma_i} \delta_{z_i} - \Gamma_i \dot{x}_i \partial_x \delta_{z_i}
  - \Gamma_i \dot{y} \partial_y \delta_{z_i} \\
  &- \Gamma_j \Gamma_i \partial_yG(z_i - z_j) \partial_x \delta_{z_i} +
  \Gamma_j \Gamma_i \partial_x G(z_i-z_j) \partial_y \delta_{z_i}= 0 
  \end{split}
  \label{eq:eom4}
\end{align}
Observe that the left hand side of \eqref{eq:eom4}
is a linear combination of the distributions
$\delta_{z_i}$, $\partial_x\delta_{z_i}$, and $\partial_y \delta_{z_i}$.
In order for \eqref{eq:eom4} to hold, the coefficients of $\delta_{z_i}$,$\partial_x \delta_{z_i}$, and $\partial_y \delta_{z_i}$ must all vanish.
This requirement yields the desired set of ordinary differential equations
\begin{align*}
  \dot{\Gamma}_i = 0 \quad,\quad
  \dot{z}_i = u(z_i)
\end{align*}
where the vector-field $u$ is given as a function of the $\Gamma_i$'s and $z_i$'s by \eqref{eq:u}.
Thus we have obtained the desired set of ordinary differential equations for the $\Gamma_i$'s and $z_i$'s.
The resulting circulation \eqref{eq:ansatz} then satisfies \eqref{eq:eom1} by construction.

\section{Derivation of the 1st order jet vortex blob method}
\label{sec:1-vortex_blob}
In this section we derive the equations of motion for the
first order vortex blob method.
This time we consider the ansatz
\begin{align}
  \omega = \sum_{i=1}^N \Gamma_i(t) \delta_{z_i(t)} + \Gamma_i^x \partial_x \delta_{z_i(t)}
  + \Gamma_i^y \partial_y \delta_{z_i(t)}
  \label{eq:1-ansatz}
\end{align} 
Our desire is to obtain an ordinary differential equation for the $\Gamma_i$'s$\Gamma_i^x$'s, $\Gamma_i^y$'s, and the $z_i$'s,
whose solutions make the $\omega$ of \eqref{eq:ansatz} a solution of \eqref{eq:eom1} and \eqref{eq:eom2}.
As in the previous section
we will suppress the time-dependence of $z_i$, $\Gamma_i$, $\Gamma^x_i$,
and $\Gamma_i^y$.

Substition of \eqref{eq:1-ansatz} into \eqref{eq:eom2} imples
\begin{align*}
  \psi = \sum_{i=1}^N \Gamma_i G_{z_i} - \Gamma^x_i \partial_xG_{z_i}
  - \Gamma^y_i \partial_yG_{z_i}
\end{align*}
where $G_{z}(\cdot) := G( \cdot - z)$ and  $G(\cdot)$ is the Green's function of $L \circ \Delta$.
By \eqref{eq:eom3} this implies, $u$, must be
\begin{align}
  u(z) = \sum_{i=1}^N \begin{pmatrix}
    \Gamma_i \partial_y G_{z_i} - \Gamma^x_i \partial_{xy}G_{z_i}
   - \Gamma^y_i \partial_{yy}G_{z_i} \\
    -\Gamma_i \partial_xG_{z_i} + \Gamma^x_i \partial_{xx}G_{z_i}
    +\Gamma^y_i \partial_{xy}G_{z_i}
  \end{pmatrix}. \label{eq:1-u}
\end{align}
Now we will compute the term ``$\pounds_u[\omega]$'' found in \eqref{eq:eom1}.
Given an arbitrary function $f \in C^\infty(\mathbb{R}^2)$ we may invoke the definition of $\pounds_u[\cdot]$ to find
\begin{align*}
  &\langle \pounds_u[ \Gamma^x_i \partial_x \delta_{z_i} ] , f \rangle = \Gamma^x_i \partial_x|_{z_i} (u^x \partial_x f + u^y \partial_y f) \\
 &\quad = \Gamma_i^x ( \partial_x u^x(z_i) \partial_xf(z_i) + u^x(z_i) \partial_{xx}f(z_i) 
 + \partial_x u^y(z_i) \partial_yf(z_i) + u^y(z_i) \partial_{xy} f(z_i) )\\
\end{align*}
In other words
\begin{align}
\begin{aligned}
  \pounds_u[\Gamma_i^x \partial_x\delta_{z_i}] &=  \Gamma_i^x ( -\partial_xu^x(z_i) \partial_x \delta_{z_i} \\
  &\quad+ u^x(z_i) \partial_{xx} \delta_{z_i}
  - \partial_x u^y(z_i) \partial_y \delta_{z_i}
  + u^y(z_i) \partial_{xy} \delta_{z_i}.
  \end{aligned}\label{eq:1-Lie}
\end{align}
A similar computation can be done to compute $\pounds_u[\Gamma^y_i \partial_y \delta_{z_i}]$.
More-over, our assumption regarding the ansatz implies
\begin{align}
\begin{aligned}
\partial_t \omega &= \dot{\Gamma}_i \delta_{z_i} - \Gamma_i \dot{x}_i \partial_x \delta_{z_i} - \Gamma_i \dot{y}_i \partial_y \delta_{z_i}\\
&\quad+\dot{\Gamma}_i^x \partial_x \delta_{z_i} - \Gamma_i^x \dot{x}_i \partial_{xx} \delta_{z_i} - \Gamma^x_i \dot{y}_i \partial_{xy} \delta_{z_i}\\
&\quad+\dot{\Gamma}^y_i \delta_{z_i} - \Gamma_i^y \dot{x}_i \partial_{xy} \delta_{z_i} - \Gamma^y_i \dot{y}_i \partial_{yy} \delta_{z_i}.
\end{aligned} \label{eq:1-time_derivative}
\end{align}
Finally, substituion of \eqref{eq:1-Lie}, and \eqref{eq:1-time_derivative} into \eqref{eq:eom1} yields
a linear combination of the distributions
$\delta_{z_i}$, $\partial_x\delta_{z_i}$,$\partial_y \delta_{z_i}$,$\partial_{xx} \delta_{z_i}$,$\partial_{xy} \delta_{z_i}$, and $\partial_{yy} \delta_{z_i}$.
In order for \eqref{eq:eom4} to hold, the coefficients of $\delta_{z_i}$,$\partial_x \delta_{z_i}$, and $\partial_y \delta_{z_i}$ must all vanish.
This requirement yields the desired set of ordinary differential equations
\begin{align*}
  \dot{z}_i &= u(z_i) \\
  \dot{\Gamma}_i &= 0 \\
  \dot{\Gamma}^x_i  &= \partial_x u^x(z_i) \Gamma^x_i + \partial_y u^x(z_i) \Gamma_i^y \\
  \dot{\Gamma}^y_i  &= \partial_x u^y(z_i) \Gamma^x_i + \partial_y u^y(z_i) \Gamma_i^y 
\end{align*}
where the vector field $u$ is given as a function of the $z_i$'s and $\Gamma$'s by \eqref{eq:1-u}.
Thus we have obtained the desired set of ordinary differential equations for the $\Gamma_i$'s and $z_i$'s.
Given a solution, the circulation \eqref{eq:1-ansatz} must neccesarily satisfy \eqref{eq:eom1}.

  It appears that the tuple $(\Gamma_i^x,\Gamma_i^y)$ is advected by the flow as a vector.  This further motivates our indexing convention.



\section{The jet-vortex blob method}
\label{sec:jet_vortex_blob}
In this section we derive the equations of motion for the
jet-vortex blob method.
Again, we start with the vorticity formulation of Euler's equations 
given in \eqref{eq:eom1}-\eqref{eq:eom3}.
Similar to the previous section, we may consider an ansatz
\begin{align}
  \omega(z) = \sum_{m+n \leq p} \sum_{i=1}^N \Gamma_i^{m,n}(t) \partial_{m,n} \delta(z-z_i(t))
  \label{eq:jet_ansatz}
\end{align}
where $\partial_{m,n} := \partial_x^m \circ \partial_y^n$,
$\Gamma_i^{m,n}(t) \in \mathbb{R}$,
and $z_i(t) \in \mathbb{R}^2$.
Our desire is to obtain an ordinary differential equation for the $\Gamma^{m,n}_i$'s and the $z_i$'s,
whose solutions make the $\omega$ of \eqref{eq:jet_ansatz} a solution of \eqref{eq:eom1} and \eqref{eq:eom2}.
To declutter future formulas,
we will suppress the time-dependence of $z_i$ and $\Gamma^{m,n}_i$
from this point onwards.

We proceed as before. Substition of \eqref{eq:jet_ansatz} into \eqref{eq:eom2} imples
\begin{align*}
  \psi = \sum_{m+n \leq p} \sum_{i=1}^N (-1)^{m+n}\Gamma_i^{m,n} \partial_{m,n} G_{z_i}
\end{align*}
By \eqref{eq:eom3} this implies, $u$, must be
\begin{align}
  u(z) = \sum_{m+n \leq p} \sum_{i=1}^N (-1)^{m+n}
  \begin{pmatrix}
    \Gamma_i^{m,n} \partial_{m,n+1} G_{z_i} \\
    -\Gamma_i^{m,n} \partial_{m+1,n} G_{z_i}
  \end{pmatrix}. \label{eq:jet_u}
\end{align}
Now we will compute the term ``$\pounds_u[\omega]$'' found in \eqref{eq:eom1}.
In appendix \ref{app:computation} we calculate
\begin{align}
  \begin{split}
  &\pounds_u[\Gamma_i^{m,n} \partial_{m,n} \delta_{z_i}] =\\
  &\Gamma_i^{m,n}
  \sum_{\ell=0}^m \sum_{k=0}^n
  (-1)^{\ell + k} \binom{n}{k} \binom{m}{\ell}
   \left(\partial_{m-\ell,n-k}u^x(z_i) \partial_{\ell+1,k} \delta_{z_i}
     +\partial_{m-\ell,n-k}u^y(z_i) \partial_{\ell,k+1} \delta_{z_i}
     \right)
   .
   \end{split}\label{eq:jet_Lie}
\end{align}
More-over, our assumption regarding the ansatz implies
\begin{align}
  \partial_t \omega = \sum_{m+n \leq p} \sum_{i=1}^N \dot{\Gamma}_i^{m,n} \delta_{z_i} - \Gamma_i^{m,n} \dot{x}_i \partial_{m+1,n} \delta_{z_i} - \Gamma_i^{m,n} \dot{y}_i \partial_{m,n+1} \delta_{z_i}. \label{eq:jet_time_derivative}
\end{align}
Substitution of \eqref{eq:jet_Lie} and \eqref{eq:jet_time_derivative} into \eqref{eq:eom1} yields
\begin{align}
  \begin{split}
    &0=\sum_{
    	\substack{
		i=1,\dots,N\\
		m+n \leq p}
		}
		\dot{\Gamma}_i^{m,n} \partial_{m,n} \delta_{z_i} - \Gamma_i^{m,n} \dot{x}_i \partial_{m+1,n} \delta_{z_i} - \Gamma_i^{m,n} \dot{y}_i \partial_{m,n+1} \delta_{z_i} \\
   & +  (-1)^{m+n} \Gamma_i^{m,n} \sum_{\ell=0}^m \sum_{k=0}^n \binom{n}{k} \binom{m}{\ell}
   \left(\partial_{m-\ell,n-k}u^x(z_i) \partial_{\ell+1,k} \delta_{z_i}
     +\partial_{m-\ell,n-k}u^y(z_i) \partial_{\ell,k+1} \delta_{z_i}
     \right)
  \end{split}
  \label{eq:eom4}
\end{align}
Observe that the left hand side of \eqref{eq:eom4}
is a linear combination of the distributions
$\partial_{m,n}\delta_{z_i}$ for $m+n \leq p+1$.
In order for \eqref{eq:eom4} to hold, the coefficients of $\partial_{m,n} \delta_{z_i}$ must vanish.
Imediately, we can observe that the coefficient of $\delta_{z_i}$ is $\dot{\Gamma}_i$.
Thus $\dot{\Gamma}_i = 0$.
Moreover, the coefficient of $\partial_{p+1,0} \delta_{z_i}$ is 
$\Gamma_i^{p,0} \dot{x}_i + \Gamma_i^{p,0} u^x(z_i)$.
Thus
\begin{align*}
  \dot{x}_i = u^x(z_i)
\end{align*}
The analogous observation for the coefficient of $\partial_{0,p+1} \delta_{z_i}$ then reveals
\begin{align*}
  \dot{y}_i = u^y(z_i)
\end{align*}

Finally, for $\ell,k < p$ the vanishing of the coefficient of $\partial_{\ell,k} \delta_{z_i}$ yields
\begin{align*}
  \dot{\Gamma}_i^{\ell,k} = (-1)^{\ell + k}
  \sum_{
    \substack{
      m > \ell \\
      n > k \\
      n+m \leq p}
    }\Gamma_i^{m,n} \Big[  \binom{n}{k} \binom{m}{\ell-1} \partial_{m-\ell+1,n-k} u^x(z_i) \\
   \binom{n}{k-1} \binom{m}{\ell} \partial_{m-\ell,n-k+1} u^y(z_i)  \Big]
\end{align*}
Upon noting that $u$ is a function of the $\Gamma$'s and $z$'s via \eqref{eq:jet_u},
we see that we have obtained the desired set of ordinary differential equations for the $\Gamma$'s and $z$'s.
The resulting circulation \eqref{eq:jet_ansatz} satisfies \eqref{eq:eom1} by construction.

\section{Conserved quantities}
\label{sec:conserved_quantities}

\subsection{Angular momentum}
\label{sec:angular_momentum}
Here are Darryl's calculations
\begin{align*}
  M &= \int \hat{z} \vec{x} \times u dx dy \\
  &= \int \vec{x} \cdot \nabla \psi dx dy \\
  &= -2 \int \psi dx dy 
\end{align*}
This is conserved because
\begin{align*}
  \frac{dM}{dt} &= \int \partial_t \psi dx dy\\
  &= \int (L\Delta)^{-1} (\partial_t \omega) dx dy \\
  &= \int (L \Delta)^{-1} ( u \cdot \nabla \omega) dx dy \\
  &= \int (L \Delta)^{-1} ( {\rm div}( u \omega) - \omega {\rm div}(u) ) dxdy \\
  &= \int {\rm div} \left( (L\Delta)^{-1}( u \omega) \right) dx dy \\
   &= 0
\end{align*}
The last line follows by the divergence theorem.

\subsection{Linear momentum}
\label{sec:linear}
The linear momentum is
\begin{align*}
  M^j = \int u^j dx dy
\end{align*}
and
\begin{align*}
  \frac{dM^j}{dt} &= \int \partial_t u^j dxdy \\
  &= \int  \partial_t (-\partial_y \psi , \partial_x \psi) dxdy \\
  &= \epsilon_{ij3} \int \partial_j( \partial_t \psi) dx dy \\
\end{align*}
Assuming $\partial_t \psi$ vanishes at $\infty$ the above integral integrates to $0$.

\subsection{Energy}
\label{sec:energy}
We will later show how \eqref{eq:eom1}-\eqref{eq:eom3} form a Hamiltonian system with respect to the Hamiltonian
\begin{align*}
  H(\omega) = \frac{1}{2} \langle \omega , G* \omega\rangle
\end{align*}
and a symplectic form which will be described in section \ref{sec:symplectic}.
Thus $H$ is conserved.
One can verify the conservation of $H$ directly.  We observe
\begin{align*}
  \frac{dH}{dt} &= \langle \partial_t \omega, G^* \omega \rangle \\
  &= \langle - \pounds_u [\omega] , G^* \omega \rangle \\
  &= \langle \omega , u^x \partial_x \psi + u^y \partial_y \psi \rangle
\end{align*}
where $\psi = G^*\omega$ and $u = \nabla^\perp \psi$.
Substituion of these identities reveals
\begin{align*}
  &= \langle \omega , (-\partial_y \psi) \partial_x \psi + (\partial_x \psi)  \partial_y \psi \rangle = 0.
\end{align*}

\section{Conservation of circulation}
\label{sec:circulation}
Let $\omega(t) \in \mathcal{D}'(\R^2)$ and let $u$ be as in \eqref{eq:eom3}.
We can consider the diffeomorphism $\varphi: \mathbb{R}^2 \to \mathbb{R}^2$
obtained by integrating the vector field $u$.
Then we can consider the pull-back $\varphi^*\omega \in \mathcal{D}'(\mathbb{R}^2)$ defined by
\begin{align*}
  \langle \varphi^* \omega , f \rangle := \langle \omega , f \circ \varphi \rangle
\end{align*}
Kelvin's circulation theorem is related to the fact that $\Omega =\varphi^* \omega$ is a constant of motion for the system.
As the $\omega$'s produced by our ansatz \eqref{eq:jet_ansatz} yield solutions to \eqref{eq:eom1}, we should expect this conservation to be inherited by the evolution of the $\Gamma^{m,n}_i$'s somehow.  In the case of $p=0$, this manifests with $\Gamma_i$ being a constant of motion.

\todo[inline]{There are more conservation laws.  This section needs to be revised, as per our Friday meeting.}

\section{Moments}
\label{sec:moments}
In this section we state the equations of motion for the moments.
The moments are given by the quantities
\begin{align*}
	M^{m,n} := \int x^m y^n  \omega .
\end{align*}
In the case where $\omega$ is of the form \eqref{eq:jet_ansatz} we observe
\begin{align*}
	M^{n,m} &=  \sum_{
		\substack{
			i=1,\dots,N\\
			\ell \leq m,k \leq n}
				}
				(-1)^{\ell + k} \Gamma_i^{\ell,k} \partial_{\ell,k}|_{z_i}(x^m y^n) \\
		&= \sum_{
		\substack{
			i=1,\dots,N\\
			\ell \leq m,k \leq n}
				}
				\Gamma_i^{\ell,k}  \frac{m! n!}{\ell! k!} x_i^{m-\ell} y_i^{n-k} 
\end{align*}
Thus
\begin{align*}
	\dot{M}^{n,m} &=  \sum_{
		\substack{
			i=1,\dots,N\\
			\ell \leq m,k \leq n}
				}
				\frac{m! n!}{\ell! k!} \Big( \dot{\Gamma}_i^{\ell,k}  x_i^{m-\ell} y_i^{n-k} 
				+  \dot{x}_i (m-\ell) \Gamma_i^{\ell,k} x_i^{m-\ell-1} y_i^{n-k}  \\
				&\hskip 20em +  \dot{y}_i (n-k) \Gamma_i^{\ell,k} x_i^{m-\ell} y_i^{n-k-1}  \Big).
\end{align*}
We observe that we have a closed system for deriving the equations of motion for the moments,
in contrast to what is typically done in moment calculations (e.g. projections).
\todo[inline]{Darryl, would you like to elaborate on this point?  You seem to think it is quite significant.
But I never really grasped the significance in our conversations.}

\section{Approximation theory}
\label{sec:approximation_theory}
In this section we will illustrate how using jet-vortices improves the accuracy of approximation of a vorticity field in a distributional sense
with respect to a reproducing kernel Hilbert space (RKHS).  Let $H(\omega) = \frac{1}{2} \langle \omega , G*\omega\rangle_{L^2} =: \frac{1}{2} \langle \omega , \omega \rangle_G$, for a Green's kernel $G:\R^2 \to \R$.
Let $h > 0$ be small and define $h \mathbb{Z}^2 = \{ (ah,bh) \in \R^2 \mid (a,b) \in \mathbb{Z}^2 \}$.
Given an $\omega \in \mathcal{D}'(\R^2)$, we can attempt to approximate $\omega$ via Dirac-deltas supported on $h \mathbb{Z}^2$.
There is a natural way to do this with respect to the Hilbert-norm because we have a reproducing kernel.
We could define $\omega_h^{(0)} = \sum_{i \in \mathbb{Z}^2} \Gamma_i \delta_{z_i}$ by requiring the error, $\omega_h^{(0)} - \omega$, to be $\langle \cdot , \cdot \rangle_{G}$-orthogonal to $\delta_z$ for each $z \in h \mathbb{Z}^2$.
This means that $G*\omega(z) =  \sum_i \Gamma_i G(z-z_i)$ for each $z \in h\mathbb{Z}^2$.
Thus $\psi_h^{(0)} = \sum_i \Gamma_i G(z-z_i)$ can be seen as a $0$th order approximation to $\psi = G*\omega$
because $\psi_h^{(0)}(z) = \psi(z)$ on the grid.
Assuming $\psi$ is continous, the error is $o(1)$ in $h$.
Moreover, $\omega^{(0)}_h$ serves as a $o(1)$ approximation to $\omega$ in the distributional sense.
Let $L$ be the psuedo-differential operator dual to $G*$.
Then $\langle \omega^{(0)}_h - \omega , f \rangle = \langle \psi - \psi_h^{(0)} , L[f] \rangle_{L^2}$,
and the sup-norm of $\psi-\psi_h^{(0)}$ is order $o(1)$, so that the whole expression is $o(1)$
for any $f$ in the RKHS produced by $G$.
 
The same reasoning applies if we consider $\omega^{(k)}_h = \sum_{i,\alpha} \Gamma_i^\alpha \partial_\alpha \delta_{z_i}$.
We define the scalars $\Gamma_i^\alpha$ via the equations
\begin{align*}
  \partial_\beta \psi (z_i) = \sum_j \Gamma_j^\alpha \partial_{\alpha\beta} G(z_i - z_j)
\end{align*}
for $\psi = G*\omega$, $z_i \in h \mathbb{Z}^2$, and $|\beta| \leq k$.
Then $\psi^{(k)}_h = \sum_{i,\alpha}\Gamma_k^\alpha \partial_\alpha G_{z_i}$
serves as a $o(h^{k})$ approximation of $\psi$ when $\psi \in C^k$
and $\omega^{(k)}_h$ serves as a $o(h^k)$ approximation of $\omega$ in
a distributional sense.

\section{Numerical Results}
\label{sec:numerics}
In this section we
\begin{enumerate}
	\item Run 0,1,2 jet vortex simulations, with Gaussian kernels
	\item Verify the $o(h^k)$ result with a log-log plot.
	\item Flow past a cylinder if we are ambitious.
\end{enumerate}

\todo[inline]{I will code this after the SHAPE conference}

\section{Hamiltonians and symplectic structures}
\label{sec:symplectic}

As a warning to the reader, the following requires some use of differential geometry
and references are provided when appropriate, as a sufficient introduction to the subject is impossible in a short space.

In the modern conception of Hamiltonian mechanics, as described in \cite{FOM,Arnold2000},
the Hamiltonian is a function on a symplectic manifold, which produces equations of motion.
An important instance of a symplectic manifold is a coadjoint orbit.
In this section we compute the coadjoint orbit of a jet-vortex
as well as the symplectic structure.

Of course, the Hamiltonian we use is given by
the kinetic energy
\begin{align*}
  H(\omega) = \frac{1}{2} \langle \omega , G* \omega \rangle.
\end{align*}
Where $\omega$ may be of the form \eqref{eq:jet_ansatz}.

Note that for a smooth vorticity $\omega \in \bigwedge^2(\R^2)$
the coadjoint orbit is
\begin{align*}
  \Orb(\omega) := \{ \varphi_* \omega \mid \varphi \in \SDiff(\R^2) \}.
\end{align*}
This may or may not be a manifold.
If $\Orb(\omega)$ is a manifold, then a vector on $\Orb(\omega)$ at the point $\tilde{\omega} \in \Orb(\omega)$ is given by $\pounds_u[\tilde\omega]$ for some (non-unique) $u \in \mathfrak{X}(\R^2)$. There is a symplectic structure on $\Orb(\omega)$
given by
\begin{align*}
  \Omega_\omega( \pounds_u[\omega] , \pounds_v[\omega] ) = \int \omega( u(x), v(x) ) \mu.
\end{align*}
where $\mu$ is a volume form on $\R^2$.
This formula is derived on page 313 of \cite{MarsdenWeinstein1983}
but it is nothing more than a special case of the KKS theorem \cite[see the boxed formula on p.303]{FOM}.
We would like to generalize it to the case of non-smooth vorticities such
as point vorticies.  To do this, we can define the function $f_\omega = \frac{\omega}{\mu}$.  Then the symplectic form takes the form
\begin{align*}
  \Omega_\omega( \pounds_u[\omega] , \pounds_v[\omega] ) = \int f_{\omega}(x) \mu( u(x), v(x) ) \mu = \langle f_{\omega}, \mu(u,v) \rangle_{L^2(\mu)}.
\end{align*}
We can identify $\omega$ with the distribution $\langle f_\omega , \cdot \rangle_{L^2(\mu)}$.
That is to say
\begin{align*}
  \langle \omega , \phi \rangle := \langle f_\omega , \phi \rangle_{L^2(\mu)}
\end{align*}
for all $\phi \in C^\infty(\R^2)$.
This allows us to let $\omega$ be something like a Dirac-delta,
or a distributional dirative of a Dirac-delta.
Moreover, the symplectic form is written as
\begin{align*}
  \Omega_\omega( \pounds_u[\omega], \pounds_v[\omega]) = \langle \omega , \mu(u,v) \rangle.
\end{align*}
In the case that $\mu = dx \wedge dy$ we can 
identify $\mu(u,v)$ with the function $u^x v^y - u^y v^x = u \times v$ so that
\begin{align*}
  \Omega_\omega( \pounds_u[\omega], \pounds_v[\omega]) = \langle \omega , u^x v^y - u^y v^x \rangle.
\end{align*}


\begin{thm}
	Let $\omega_0 = \gamma_i^\alpha \partial_{\alpha} \delta_{Z_i}$ for $\gamma_i^\alpha \neq 0$ and let
	$M^{(k)} = \{ \sum_{|\alpha| \leq k} \Gamma_i^\alpha \partial_{\alpha} \delta_{Z_i} \}$.
	Then $\Orb(\omega_0)$ is a sub-manifold of the $M^{(k)}$.
\end{thm}
\begin{proof}
	Let $\varphi \in \SDiff(\R^2)$.  Then
	\begin{align*}
		\langle \varphi_* \omega_0 , f \rangle &= \langle \omega_0 , f \circ \varphi \rangle \\
                &= \gamma_i^\alpha \partial_{\alpha}|_{z=Z_i} (f \circ\varphi)(z) \\
                &= \gamma_i^\alpha \partial_{\beta}|_{\varphi(Z_i)} f \partial_\beta|_{Z_i} \varphi^\alpha.
	\end{align*}
        Where we have used the Faa di Bruno formula via the multi-indexing convention of \cite{Jacobs2014b}.
        Thus $\varphi_* \omega_0 \in M^{(k)}$.
\end{proof}

If we would like a more geometric description of what $M^{(k)}$ is,
we could identify $M^{(k)}$ as a dual vector-bundle
\begin{align*}
  M^{(k)} = \left(\Jet^{(k)}( \R^2 , \R)^n \right)^*
\end{align*}
where we view $\Jet^{(k)}( \R^2 , \R)$ as a vector-bundle over $\R^2$.
This identification is tautological upon noting that for $\omega \in M^{(k)}$
the quantity $ \langle \omega , f \rangle$
is a linear function of the $k$th order Taylor-coefficients of $f \in C^\infty(\R^2)$
about some points $z_1,\dots,z_n$.

Finally, as a sanity check, we can also prove that the equations of motion derived previously are identical to
the equations of motion obtained using this symplectic structure.

\begin{thm}
	Let $H(\omega) = \frac{1}{2} \langle \omega , G*\omega \rangle_{L^2}$.  Then Hamilton's equations on $\Orb(\omega_0)$
	are given by
	\begin{align*}
		\partial_t \omega + \pounds_u[\Omega] \quad,\quad u = \nabla^\perp \psi \quad,\quad \psi = G*\omega.
	\end{align*}
\end{thm}
\begin{proof}
  Let $\omega \in \Orb(\omega_0)$, and let $X_H(\omega) = \pounds_u[\omega]$ for some (non-unique) $u \in \mathfrak{X}_{\rm div}(\R^2)$.
  Out goal is to solve for $u$.
  By the definition of $X_H$ we see that for any
  $v \in \mathfrak{X}_{\rm div}(\R^2)$
  \begin{align*}
    \langle \omega , u^x v^y - u^y v^x \rangle &=
    \Omega_{\omega}( \pounds_u[\omega] , \pounds_v[\omega] ) = \langle dH(\omega) , \pounds_v[\omega] \rangle \\
    &= \langle G * \omega , \pounds_v[\omega] \rangle 
  \end{align*}
  If we let $\psi := G*\omega$ then
  \begin{align*}
    = - \langle \omega , \pounds_{v}[\psi] \rangle 
    = \langle \omega , v^x \partial_x \psi + v^y \partial_y \psi \rangle
  \end{align*}
  We see that $u = (-\partial_y \psi,\partial_x \psi)$
  would be a possible solution.
  As $\Omega$ is non-degenerate, this is the unique solution.
\end{proof}

\subsection{The first order case}
Let us deal with the first order case.
\begin{thm}
Let $Z_1,\dots,Z_n \in \R^2$ be distinct.
The coadjoint orbit of
\begin{align*}
  \omega = \sum_{i=1}^N \gamma_i \delta_{Z_i} + \gamma_i^x \partial_x \delta_{Z_i} + \gamma_i^y \partial_{y} \delta_{z_i}
\end{align*}
is
\begin{align*}
  \Orb(\omega) &= \left\{ \sum_{i=1}^n \gamma_i \delta_{z_i} + \Gamma_i^x \partial_x \delta_{\tilde{z}_i} + \Gamma_i^y \partial_{y} \delta_{z_i}
  \mid z_i \in \R^2, (\Gamma_i^x,\Gamma_i^y) \in \R^2 \backslash \{0\} \right\} \\
  &\cong  \{ (z_1,\dots,z_n,\Gamma_1,\dots,\Gamma_n) \mid z_i \in \R^2, \Gamma_i \in \R^2 \backslash \{0\} , ( i \neq j \implies z_i \neq z_j ) \}.
\end{align*}
The symplectic structure is
  \begin{align*}
    \Omega( (\dot{z},\dot{\Gamma}), (\delta z,\delta \Gamma) ) &=
    \gamma_i ( \dot{x}_i \cdot \delta y_i - \delta x_i \cdot \dot{y}_i ) \\
    &\quad + \dot{\Gamma}_i^x \cdot \delta y_i
    - \dot{\Gamma}_i^y \cdot \delta x_i
    + \delta \Gamma_i^y \cdot \dot{x}_i 
    - \delta \Gamma_i^x \cdot \dot{y}_i
  \end{align*}
\end{thm}
\begin{proof}
  Let $\omega$ be as above and consider some $\varphi \in \SDiff(\R^2)$.
  We find that for any function $f$
  \begin{align*}
    \langle \varphi_* \omega \mid f \rangle := \langle \omega , \varphi^* f\rangle
    &= \gamma_i f(\varphi(Z_i)) \\
    &\quad - \gamma_i^x \partial_x \varphi^x|_{z=Z_i} \partial_x f |_{z=\varphi(Z_i)}
    - \gamma_i^x \partial_x \varphi^y|_{z=Z_i} \partial_y f |_{z=\varphi(Z_i)} \\
    &\quad - \gamma_i^y \partial_y \varphi^x|_{z=Z_i} \partial_x f |_{z=\varphi(Z_i)}
    - \gamma_i^y \partial_y \varphi^y|_{z=Z_i} \partial_y f |_{z=\varphi(Z_i)}
  \end{align*}
  Collecting like terms we find
  \begin{align*}
    \varphi_* \omega = \gamma_i \delta_{\varphi(Z_i)} + \Gamma_i^x \partial_x \delta_{\varphi(Z_i)} + \Gamma_i^y \partial_y \delta_{\varphi(Z_i)}
  \end{align*}
  where
  \begin{align*}
    \Gamma = 
    \begin{bmatrix}
      \Gamma_i^x \\ \Gamma_i^y 
    \end{bmatrix}
    =
    D\varphi(Z_i) \cdot
    \begin{bmatrix}
      \gamma_i^x \\ \gamma_i^y
    \end{bmatrix}
  \end{align*}
  By varying $\varphi$ we can obtain 
  any collection of distinct points $z_1,\dots,z_n \in \R^2$
  and any collection of non-zero vectors $\Gamma_1,\dots,\Gamma_n \in \R^2 \backslash \{0\}$.
  This proves the first claim.

  To derive the symplectic structure recall the symplectic structure for a general vorticity.
  \begin{align}
    \Omega( \pounds_u[\omega] , \pounds_v[\omega] )
    = \langle \omega , u \wedge v \rangle \label{eq:vorticity_symplectic_form}
  \end{align}
  where we view $\omega$ as a distribution and $u \wedge v = u^x v^y - v^x u^y$
  as a real-valued function.
  Now let $\omega = \gamma_i \delta_{z_i} + \Gamma_i^x \partial_x\delta_{z_i} + \Gamma_i^y \partial_y \delta_{z_i}$.
  In this case the right hand side can be computed to be
  \begin{align*}
    \langle \omega , u \wedge v \rangle &= \gamma_i ( u^x(z_i)v^y(z_i) - v^x(z_i) u^y(z_i) ) \\
    &\quad + \Gamma_i^x ( u^x_{,x} v^y + u^x v^y_{,x} - v^x_{,x}u^y - v^x u^y_{,x})|_{z = z_i} \\
    &\quad + \Gamma_i^y ( u^x_{,y} v^y + u^x v^y_{,y} - v^x_{,y}u^y - v^x u^y_{,y})|_{z = z_i}
  \end{align*}
  Note that this is writen entirely in terms of the 1-jet of $u$ and $v$ evaluated 
  at $z_i$.
  Moreover, $\pounds_u[\omega] = \gamma_0 u^x(z_i) \partial_{z_i} + \dots$
  also has the property that it is only dependent on
  the one-jet of $u$ and $v$ at the poinst $z_1,\dots,z_n$.
  Therefore both side of \eqref{eq:vorticity_symplectic_form}
  can be written as a function of the finite collection of 
  numbers $u(z_i), Du(z_i), v(z_i),Dv(z_i)$.
  The result then follows by noting
  \begin{align*}
    u(z_i) \mapsto u_{z_i}\\
    u^x_{,x}(z_i) \Gamma^x_i + u^x_{,y}(z_i) \Gamma^y_i \mapsto \dot{\Gamma}_i^x \\
    u^y_{,x}(z_i) \Gamma^x_i + u^y_{,y}(z_i) \Gamma^y_i \mapsto \dot{\Gamma}_i^y.
    \end{align*}
    under the tangent lift of the coordinate chart
    \begin{align*}
      \sum_i \gamma_i \delta_{z_i} + \Gamma_i^x \partial_x \delta_{z_i} + \Gamma_i^y \partial_y \delta_{z_i} \mapsto (z_1,\dots,z_n,\Gamma_1,\dots,\Gamma_n)
      \end{align*}
\end{proof}


\section{Conclusion}

\todo[inline]{Darryl should take a stab at this first.}


\appendix
\section{Computations}
\label{app:computation}

Here we compute $\pounds_u [\partial_{m,n} \delta_z]$.
We begin with a lemma
\begin{lem}
  Let $f,g \in \mathbb{C}^\infty(\R^2)$.  Then
  \begin{align*}
    \partial_x^m ( fg) = \sum_{\ell=0}^{m} \binom{m}{\ell} (\partial_x^\ell f)(\partial_x^{m-\ell}g).
  \end{align*}
\end{lem}
\begin{proof}
  For $m=1$ the formula is the standard product rule.
  Assume the formula holds at order $m$.
  Then we find
  \begin{align*}
    \partial_x^{m+1} (fg) &= \partial_x ( \partial_x^{m}(fg)) \\
    &= \partial_x \left( \sum_{\ell=0}^{m} \binom{m}{\ell} (\partial_x^\ell f)(\partial_x^{m-\ell}g) \right) \\
    &= \sum_{\ell=0}^{m} \binom{m}{\ell}\left(
      ( \partial_x^{\ell + 1} f) (\partial_x^{m-\ell}g) 
      + ( \partial_x^{\ell} f) (\partial_x^{m-\ell+1}g)
      \right) \\
     &= f (\partial_x^{m+1}g) +(\partial_x^{m+1}f) g + \sum_{\ell=1}^{m} \left( \binom{m}{\ell} + \binom{m}{\ell-1} \right) (\partial_x^\ell f)(\partial_x^{m-\ell}g) \\
     &= \sum_{\ell=0}^{m+1} \binom{m+1}{\ell} (\partial_x^\ell f)(\partial_x^{m-\ell}g).
  \end{align*}
  The lemma follows by induction.
\end{proof}

We can now compute $\pounds_u[\partial_{m,n} \delta_z]$.
Let $f \in C^\infty(\R^2)$.  We find
\begin{align*}
  \langle \pounds_u [\partial_{m,n} \delta_z] , f \rangle 
  &= -\langle \partial_{m,n} \delta_z , u^x \partial_x f + u^y \partial_y f \rangle \\
  &= -(-1)^{m+n} \partial_{m,n}|_{z} \left( u^x \partial_x f + u^y \partial_y f \right)
\end{align*}
By two applications of the lemma we find
\begin{align*}
  &= -(-1)^{m+n} \partial_x^m|_z \left( \sum_{k=0}^{n} \binom{n}{k}
    \left(\partial_{0,k} u^x \cdot \partial_{1,n-k} f + \partial_{0,k} u^y \cdot \partial_{0,n-k+1} f \right)
    \right) \\
  &= -(-1)^{m+n} \sum_{\ell=0}^{m} \sum_{k=0}^{n} \binom{n}{k}\binom{m}{\ell}
    \Big(\partial_{\ell,k} u^x(z) \partial_{m-\ell+1,n-k} f(z) \\
     &\hskip 16em + \partial_{\ell,k} u^y(z) \partial_{m-\ell,n-k+1} f(z)
    \Big)
\end{align*}
Therefore we find
\begin{align*}
  &\pounds_u[ \partial_{m,n} \delta_{z_i}] =\\
  &=\quad (-1)^{m+n} \sum_{\ell=0}^m \sum_{k=0}^n \binom{n}{k} \binom{m}{\ell} (-1)^{m+n-\ell-k}\\
  & \qquad \qquad \cdot \left(\partial_{\ell,k}u^x(z_i) \partial_{m-\ell+1,n-k} \delta_{z_i}
     +\partial_{\ell,k}u^y(z_i) \partial_{m-\ell,n-k+1} \delta_{z_i}
     \right) \\
  &=\sum_{\ell=0}^m \sum_{k=0}^n (-1)^{\ell + k}\binom{n}{k} \binom{m}{\ell}
   \left(\partial_{m-\ell,n-k}u^x(z_i) \partial_{\ell+1,k} \delta_{z_i}
     +\partial_{m-\ell,n-k}u^y(z_i) \partial_{\ell,k+1} \delta_{z_i}
     \right) 
\end{align*}

\bibliographystyle{amsalpha}
\bibliography{/Users/hoj201/Dropbox/hoj_2014.bib}
\end{document}
