\documentclass[12pt]{amsart}
\usepackage{amsmath,amssymb}
\usepackage{geometry} % see geometry.pdf on how to lay out the page. There's lots.
\geometry{a4paper} % or letter or a5paper or ... etc
% \geometry{landscape} % rotated page geometry

%  POSSIBLY USEFULE PACKAGES
%\usepackage{graphicx}
%\usepackage{tensor}
\usepackage{todonotes}

%  NEW COMMANDS
\newcommand{\pder}[2]{\ensuremath{\frac{ \partial #1}{\partial #2}}}
\newcommand{\ppder}[3]{\ensuremath{\frac{\partial^2 #1}{\partial
      #2 \partial #3} } }

%  NEW THEOREM ENVIRONMENTS
\newtheorem{thm}{Theorem}[section]
\newtheorem{prop}[thm]{Proposition}
\newtheorem{cor}[thm]{Corollary}
\newtheorem{lem}[thm]{Lemma}
\newtheorem{defn}[thm]{Definition}


%  MATH OPERATORS
\DeclareMathOperator{\Diff}{Diff}
\DeclareMathOperator{\GL}{GL}
\DeclareMathOperator{\SO}{SO}
\DeclareMathOperator{\ad}{ad}
\DeclareMathOperator{\Ad}{Ad}

%  TITLE, AUTHOR, DATE
\title{Jet vortex methods}
\author{Darryl D. Holm, Henry O. Jacobs}
\date{\today}


\begin{document}

\maketitle

\begin{abstract}
  blah blah blah.
\end{abstract}

\section{Introduction}
\label{sec:intro}


\section{Notation}
The symbol ``$z$'' will be used to denote
an element of $\mathbb{R}^2$
with coordinates $z = (x,y)$.
The Dirac-delta distribution on $\mathbb{R}^2$
will be denoted by $\delta(\cdot)$, and $\delta_z(\cdot) := \delta( \cdot - z)$.
The Laplace operator will be denoted by $\Delta := \partial_x^2 + \partial_y^2$.

\section{Background}

\subsection{Vortex methods}

\subsection{Regularized fluid modelling and turbulence}

\todo[inline]{In this section we should relate regularized Euler equations 
to turbulence modelling, and introduce a TRI differential operator $L$
which yields the regularized model $\partial_t m + \pounds_u [m] = 0, m = L[u]$.}

\section{Derivation of the vortex blob method}
\label{sec:vortex_blob}
In this section we derive the equations of motion for the traditional
vortex blob method \todo{cite stuff}.
We can begin by considering the vorticity formulation of Euler's equations for an incompressible inviscid planar fluid
\begin{align}
  &\partial_t \omega + u^1 \partial_x \omega + u^2 \partial_x \omega = 0 \label{eq:eom1} \\
  &\omega = (L \circ \Delta) \cdot \psi \label{eq:eom2} \\
  &u = (u^1,u^2) = \nabla^\perp \psi :=  (- \partial_y \psi, \partial_x \psi)
  \label{eq:u}
\end{align}
By viewing the vorticity, $\omega$, is as a distribution,
we may consider the ansatz
\begin{align}
  \omega(z) = \Gamma_i(t) \delta(z-z_i(t))
  \label{eq:ansatz}
\end{align} 
where we have invoked the Einstein summation convention to sum over 
the index $i=1,\dots,n$,
with $\Gamma_i(t) \in \mathbb{R}$,
and $z_i(t) \in \mathbb{R}^2$.
Our desire is to obtain an ordinary differential equation for the $\Gamma_i$'s and the $z_i$'s,
whose solutions make the $\omega$ of \eqref{eq:ansatz} a solution of \eqref{eq:eom1} and \eqref{eq:eom2}.
To declutter future formulas,
we will suppress the time-dependence of $z_i$ and $\Gamma_i$
from this point onwards.

In order to find the desired ordinary differential equation for the $\Gamma_i$'s and $z_i$'s we will seek consistency constraints implied by the equations of motion \eqref{eq:eom1},\eqref{eq:eom2}.
Substition of \eqref{eq:ansatz} into \eqref{eq:eom2} imples
\begin{align*}
  \psi = \Gamma_i G_{z_i}
\end{align*}
where $G_{z}(\cdot) := G( \cdot - z)$ and  $G(\cdot)$ is the Green's function of $L \circ \Delta$.
This implies, $u$, must be
\begin{align*}
  u(z) = \begin{pmatrix}
   - \Gamma_i \partial_y G_{z_i} \\
    \Gamma_i \partial_x G_{z_i}
  \end{pmatrix}.
\end{align*}
Finally, substituion of the above velocity field
and the ansatz \eqref{eq:ansatz} into \eqref{eq:eom1} yields
\begin{align}
  \partial_t ( \Gamma_i \delta_{z_i}) +  (- \Gamma_j \partial_yG_{z_j}) \cdot \partial_x ( \Gamma_i \delta_{z_i})  +
  (\Gamma_j \partial_x G_{z_j} ) \cdot \partial_x ( \Gamma_i \delta_{z_i})= 0 \label{eq:eom3}
\end{align}
Upon writing \eqref{eq:eom3} in terms of distributional diratives we find
\begin{align}
  \begin{split}
  &\dot{\Gamma_i} \delta_{z_i} - \Gamma_i \dot{x}_i \partial_x \delta_{z_i}
  - \Gamma_i \dot{y} \partial_y \delta_{z_i} \\
  &- \Gamma_j \Gamma_i \partial_yG(z_i - z_j) \partial_x \delta_{z_i}  +
  \Gamma_j \Gamma_i \partial_x G(z_i-z_j) \partial_y \delta_{z_i}= 0 
  \end{split}
  \label{eq:eom4}
\end{align}
Observe that the left hand side of \eqref{eq:eom4}
is a linear combination of the distributions
$\delta_{z_i}$, $\partial_x\delta_{z_i}$, and $\partial_y \delta_{z_i}$.
In order for \eqref{eq:eom4} to hold, the coefficients of $\delta_{z_i}$,$\partial_x \delta_{z_i}$, and $\partial_y \delta_{z_i}$ must all vanish.
This requirement yields the desired set of ordinary differential equations
\begin{align*}
  \dot{\Gamma}_i = 0 \quad,\quad
  \dot{x}_i = - \Gamma_j \partial_y G(z_i - z_j) \quad,\quad
  \dot{y}_i =  \Gamma_j \partial_x G(z_i - z_j).
\end{align*}

\todo[inline]{Write a few ending sentences.  Section should not end with an equation}

\section{The jet-vortex blob method}
\label{sec:jet_vortex_blob}

\todo[inline]{This section should mimic the previous,
almost word-for-word}

In this section we derive the equations of motion for the
jet-vortex blob method.
As before, we begin with the vorticity equation \eqref{eq:eom1}-\eqref{eq:u}and consider the ansatz
\begin{align}
  \omega(z) = \Gamma_i^\alpha(t) \partial_\alpha \delta(z-z_i(t))
  \label{eq:jet_ansatz}
\end{align} 
where we have invoked the Einstein summation convention to sum over 
the index $i=1,\dots,n$ and the multi-index $\alpha$ of rank $|\alpha| < m$,
As before, our desire is to obtain an ordinary differential equation for the $\Gamma^\alpha_i$'s and the $z_i$'s,
whose solutions make the $\omega$ of \eqref{eq:jet_ansatz} a solution of \eqref{eq:eom1}-\eqref{eq:u}.
As before,
we will suppress the time-dependence of $z_i$ and $\Gamma^\alpha_i$.

Substition of \eqref{eq:jet_ansatz} into \eqref{eq:eom2} imples
\begin{align*}
  \psi = \Gamma^\alpha_i \partial_\alpha G_{z_i}
\end{align*}
where $G_{z}(\cdot) := G( \cdot - z)$ and  $G(\cdot)$ is the Green's function of $L \circ \Delta$.
This implies, $u$, must be
\begin{align*}
  u = \begin{pmatrix}
   - \Gamma_i^\alpha \partial_{y \cup \alpha} G_{z_i} \\
    \Gamma_i^\alpha \partial_{x \cup \alpha} G_{z_i}
  \end{pmatrix}.
\end{align*}
Finally, substituion of the above velocity field
and the ansatz \eqref{eq:jet_ansatz} into \eqref{eq:eom1} yields
\begin{align}
  \partial_t ( \Gamma^\alpha_i \partial_\alpha \delta_{z_i}) +  (- \Gamma^\beta_j \partial_{y \beta} G_{z_j}) \cdot \partial_x ( \Gamma^\alpha_i \partial_\alpha \delta_{z_i})  +
  (\Gamma_j^\beta \partial_{x \beta} G_{z_j} ) \cdot \partial_x  ( \Gamma^\alpha_i \partial_\alpha \delta_{z_i})= 0 \label{eq:eom3}
\end{align}
Upon writing \eqref{eq:eom3} in terms of distributional diratives we find
\begin{align}
  \begin{split}
  &\dot{\Gamma_i} \delta_{z_i} - \Gamma_i \dot{x}_i \partial_x \delta_{z_i}
  - \Gamma_i \dot{y} \partial_y \delta_{z_i} \\
  &- \Gamma_j \Gamma_i \partial_yG(z_i - z_j) \partial_x \delta_{z_i}  +
  \Gamma_j \Gamma_i \partial_x G(z_i-z_j) \partial_y \delta_{z_i}= 0 
  \end{split}
  \label{eq:eom4}
\end{align}
Observe that the left hand side of \eqref{eq:eom4}
is a linear combination of the distributions
$\delta_{z_i}$, $\partial_x\delta_{z_i}$, and $\partial_y \delta_{z_i}$.
In order for \eqref{eq:eom4} to hold, the coefficients of $\delta_{z_i}$,$\partial_x \delta_{z_i}$, and $\partial_y \delta_{z_i}$ must all vanish.
This requirement yields the desired set of ordinary differential equations
\begin{align*}
  \dot{\Gamma}_i = 0 \quad,\quad
  \dot{x}_i = - \Gamma_j \partial_y G(z_i - z_j) \quad,\quad
  \dot{y}_i =  \Gamma_j \partial_x G(z_i - z_j).
\end{align*}

\todo[inline]{Write a few ending sentences.  Section should not end with an equation}


\section{Conserved quantities}
\label{sec:conserved_quantities}

\todo[inline]{Linear momentum, angular momentum, and energy}

\section{Conservation of circulation}
\label{sec:circulation}


\section{Moments}
\label{sec:moments}


\section{Approximation theory}
\label{sec:approximation_theory}

We have order $o(h^k)$ approximation.

\section{Hamiltonians and symplectic structures}
\label{sec:symplectic}


\section{Conclusion}

\appendix
\section{Multi-index conventions}
\label{app:multi}

\bibliographystyle{amsalpha}
\bibliography{/Users/hoj201/Dropbox/hoj_2014.bib}
\end{document}
