\documentclass[12pt]{amsart}
\usepackage{amsmath,amssymb}
\usepackage{geometry} % see geometry.pdf on how to lay out the page. There's lots.
\geometry{a4paper} % or letter or a5paper or ... etc
% \geometry{landscape} % rotated page geometry

%  POSSIBLY USEFULE PACKAGES
%\usepackage{graphicx}
%\usepackage{tensor}
\usepackage{todonotes}

%  NEW COMMANDS
\newcommand{\pder}[2]{\ensuremath{\frac{ \partial #1}{\partial #2}}}
\newcommand{\ppder}[3]{\ensuremath{\frac{\partial^2 #1}{\partial
      #2 \partial #3} } }
\newcommand{\R}{\ensuremath{\mathbb{R}}}

%  NEW THEOREM ENVIRONMENTS
\newtheorem{thm}{Theorem}[section]
\newtheorem{prop}[thm]{Proposition}
\newtheorem{cor}[thm]{Corollary}
\newtheorem{lem}[thm]{Lemma}
\newtheorem{defn}[thm]{Definition}
\newtheorem{rmk}[thm]{Remark}

%  MATH OPERATORS
\DeclareMathOperator{\SDiff}{SDiff}
\DeclareMathOperator{\Emb}{Emb}
\DeclareMathOperator{\Jet}{Jet}
\DeclareMathOperator{\SO}{SO}
\DeclareMathOperator{\ad}{ad}
\DeclareMathOperator{\Ad}{Ad}
\DeclareMathOperator{\Orb}{Orb}

%  TITLE, AUTHOR, DATE
\title{Notes on jet vorticies}
\author{Henry O. Jacobs}
\date{\today}


\begin{document}

\maketitle

\begin{abstract}
  A derivation of the equations of motion for the jet vortex method
  and its symplectic structure, as well as some approximation theory.
\end{abstract}

\section{Outline}

Vortex blob is awesome.
Vortex blob links to regularized fluid modelling (turbulence).
We consider a generalization wherein the solution ansatz is (...).

Comparison with moments (moment approximation does not close, so projection is neccessary).  We don't need to worry about closure.

\begin{enumerate}
\item Vortex blob
\item EOM for k-vorts
\item Circulation
\item Ang+lin+energy
\item Moments
\item Higher order accuracy
\item Symplecticity
\end{enumerate}


\section{Angular momentum}
\label{sec:angular_momentum}
Here are Darryl's calculations
\begin{align*}
  M &= \int \hat{z} \vec{x} \times u dx dy \\
  &= \int \vec{x} \cdot \nabla \psi dx dy \\
  &= -2 \int \psi dx dy 
\end{align*}
This is conserved because
\begin{align*}
  \frac{dM}{dt} &= \int \partial_t \psi dx dy\\
  &= \int (L\Delta)^{-1} (\partial_t \omega) dx dy \\
  &= \int (L \Delta)^{-1} ( u \cdot \nabla \omega) dx dy \\
  &= \int (L \Delta)^{-1} ( {\rm div}( u \omega) - \omega {\rm div}(u) ) dxdy \\
  &= \int {\rm div} \left( (L\Delta)^{-1}( u \omega) \right) dx dy \\
   &= 0
\end{align*}
The last line follows by the divergence theorem.

\section{Linear momentum}
\label{sec:linear}
Linear momentum is
\begin{align*}
  M^j = \int u^j dx dy
\end{align*}
and
\begin{align*}
  \frac{dM^j}{dt} &= \int \partial_t u^j dxdy \\
  &= \int (\hat{z} \times \nabla) \partial_t \psi dxdy \\
  &= \epsilon_{ij3} \int \partial_j( \partial_t \psi) dx dy \\
\end{align*}
... goes to zero by integration by parts.



\section{List of accomplishments}
So far we have accomplished the following
\begin{enumerate}
\item We found the equations of motion.
\item We found a geometric description of the coadjoint orbits as submanifold of $M^{(k)} = [\Jet^{(k)}(\R^2,\R)^n]^*$.
\item We found the symplectic structure on each orbit by invoking \cite{MarsdenWeinstein1983}.
\item We found that the vorticity field can be approximated to $o(h^k)$ on a grid of $k$th order jet vortices.
\end{enumerate}

Extra things we could do are
\begin{enumerate}
\item Make some movies (Stefan's ``fluidx'' code is CUDA enabled)
\item Calculate the improvement in computational efficiency obtained by using a smaller number of jet-vortices over a large number of vortices.
  There would be fewer pair-wise interactions.
\item Kelvin's theorem
\item other conservation laws
\item Boundary condition (method of images)
\end{enumerate}


\section{The equations of motion}
\label{sec:eom}
Let $\omega$ be a time-dependent distribution and consider the evolution equation
\begin{align}
  \frac{d\omega}{dt} + \pounds_u [\omega] = 0 \label{eq:eom}
\end{align}
where $u = \nabla^\perp \psi$ and $\omega = \Delta ( L[ \psi] )$ where $\psi \in [C^2(\R^2) / \R]$ is the stream function of $u$ and $L$ is some
${\rm SE}(2)$ invariant psuedo-differential operator.
Let $K \in C^2(\R^2)$ be the kernel of $\Delta \circ L$.

If we consider the anstaz
\begin{align*}
  \omega = \sum_{|\alpha| \leq k} \Gamma_i^\alpha \partial_\alpha \delta_{z_i}
\end{align*}
We find that
\begin{align*}
  \partial_t \omega = \sum_{|\alpha| \leq k} \frac{d\Gamma_i^\alpha}{dt} \partial_{\alpha}\delta_{z_i} - \Gamma_i^{\alpha}\dot{x}_i \partial_{\alpha x}\delta_{z_i} 
  - \Gamma_i^{\alpha} \dot{y}_i \partial_{\alpha y} \delta_{z_i}
\end{align*}
and
\begin{align*}
  \pounds_u[\omega] = \sum_{|\alpha|\leq k}(-1)^{|\alpha|} \Gamma_i^\alpha  
  \sum_{ [\alpha_1,\alpha_2] \in \Pi(\alpha,2)}
  (-1)^{|\alpha_2|} \left(
  \partial_{\alpha_1}u^x(z_i) \partial_{\alpha_2 x} \delta_{z_i} +
  \partial_{\alpha_1}u^y(z_i) \partial_{\alpha_2 y} \delta_{z_i}
  \right)
\end{align*}
We notice that both $\partial_t \omega$ and $\pounds_u [\omega]$
are sums of $\partial_{\beta} \delta_{z_i}$ for $|\beta| \leq k$.
Therefore the space of $\omega$'s of this form is invariant under
the evolution of \eqref{eq:eom}.
Due to this invariance, we may restrict to this finite-dimensional submanifold, and calculate an ODE.

\begin{rmk}
  The coefficient of  $\delta_{z_i}$
  in $\partial_t \omega$ is $\frac{d\Gamma_i}{dt}$,
  while the coefficient of $\delta_{z_i}$ in $\pounds_u[\omega]$ is $0$.
  Therefore, if $\partial_t \omega + \pounds_{u}[\omega] = 0$
  it must be the case that $\frac{d\Gamma_i}{dt} = 0$.
\end{rmk}
\begin{rmk}
  The coefficient of $\partial_{\alpha x} \delta_{z_i}$ for $|\alpha|=k$
  in $\partial_t \omega + \pounds_u[\omega]$ is
  $- \dot{x}_i \Gamma_i^\alpha + \Gamma_i^\alpha u^x(z_i)$.
  If $\Gamma_i^\alpha \neq 0$ then $\dot{x}_i = u^x(z_i)$.
  Similarly we would find $\dot{y}_i = u^y(z_i)$ by checking the coefficient of $\partial_{\alpha y} \delta_{z_i}$.
\end{rmk}

\subsection{The first order equations}
Here we consider the ansatz $\omega = \Gamma_i \delta_{z_i} + \Gamma^{x}_i \partial_x \delta_{z_i} + \Gamma_i^y \partial_y \delta_{z_i}$.
Then we find
\begin{align*}
  \psi = \Gamma_i K_{z_i} - \Gamma^x_i \partial_x K_{z_i} - \Gamma^y_i \partial_y K_{z_i}
\end{align*}
where $K_{z_i}(z) := K(z-z_i)$.
To calculate the dynamics we find that
\begin{align*}
  \partial_t \omega = \dot{\Gamma}_i \delta_{z_i} -  \sum_{\alpha \in \{x,y\}} \Gamma_i^\alpha \dot{z}_i^{\alpha} \partial_\alpha \delta_{z_i} + \dot{\Gamma}^\alpha_i \partial_\alpha \delta_{z_i} -  \sum_{\beta \in \{x,y\}} \Gamma^\alpha_i \dot{z}_i^{\beta} \partial_{\alpha \beta} \delta_{z_i}
\end{align*}
and
\begin{align*}
  \pounds_u [\omega] = \Gamma_i \pounds_u[\delta_{z_i}] + \Gamma_i^\alpha \pounds_u[ \partial_\alpha \delta_{z_i}].
\end{align*}
To make this more explicit, we shoud figure out what distributions $\pounds_u[\delta_{z}]$ and $\pounds_u[\partial_\alpha \delta_{z}]$ are in terms of familar distributions.  If $f \in C^{\infty}(\R^2)$, we find
\begin{align*}
  \langle \pounds_u[\delta_{z}] \mid f \rangle := - \langle \delta_{z} \mid \pounds_u[f] \rangle = - \langle \delta_{z} \mid u^\alpha f_{,\alpha} \rangle = - u^\alpha(z) f_{,\alpha}(z).
\end{align*}
so $\pounds_u[\delta_{z}] = u^\alpha(z) \partial_\alpha \delta_{z}$.
Similarly we find
\begin{align*}
  \langle \pounds_u [ \partial_\alpha \delta_{z}]  \mid f \rangle
  &:= - \langle \partial_\alpha \delta_{z} \mid \pounds_u[f] \rangle
  = - \langle \partial_\alpha \delta_{z} \mid u^\alpha  f_{,\alpha} \rangle 
  = \partial_\alpha |_{z} ( u^\beta f_{,\beta})\\
  &= u^\beta_{,\alpha}(z) f_{,\beta}(z) + u^\beta(z) f_{,\alpha\beta}(z).
\end{align*}
Thus $\pounds_u[\partial_\alpha \delta_{z}] = -  \partial_\alpha u^\beta(z_i) \partial_\beta \delta_{z_i} + u^\beta(z_i) \partial_{\alpha\beta} \delta_{z_i}$.
Substituting these identities into \eqref{eq:eom} we find that
\begin{align*}
  \dot{z}_i &= u(z_i) \\
  \dot{\Gamma}_i &= 0 \\
  \dot{\Gamma}^\alpha_i &= \Gamma_i^\beta u^\alpha_{,\beta}(z_i)
\end{align*}

% \section{Second order}
% Continuing the pattern let
% \begin{align*}
%   \omega = p_\alpha \delta_{q^\alpha} + d^i_\alpha \partial_i \delta_{q^\alpha} + Q^{ij}_\alpha \partial_{ij} \delta_{q^\alpha}
% \end{align*}
% so that the stream function must be
% \begin{align*}
%   \psi = p_\alpha K_{q^\alpha} - d^i_\alpha \partial_i K_{q^\alpha} + Q^{ij}_\alpha \partial_{ij} K_{q^\alpha}
% \end{align*}
%  and $u = \nabla^\perp \psi$.
%  We compute
% \begin{align*}
%   \langle \pounds_u[ \partial_{ij} \delta_{q}] , f \rangle
%   &= - \langle \partial_{ij} \delta_{q} , u^k \partial_k f \rangle 
%   = - \partial_{ij} |_{x = q} ( u^k(x) \partial_k f(x) ) \\
%   &= - ( u^k_{,ij} f_{,k} + u^k_{,j} f_{,ik} + u^k_{,i} f_{,jk} + u^k f_{,ijk} )(q) 
% \end{align*}
% Thus
% \begin{align*}
%   \pounds_{u} [ \partial_{ij} \delta_q] =  u^k_{,ij}(q) \partial_k \delta_q - u^k_{,j}(q) \partial_{ik} \delta_q - u^k_{,i}(q) \partial_{jk}\delta_q  + u^k(q) \partial_{ijk} \delta_q 
% \end{align*}

% \subsection{Finite order equation}

\section{Coadjoint orbits and symplectic structures}
\label{sec:orbits}

In this section we compute the coadjoint orbit of a jet-vortex
as well as the symplectic structure.

Note that for a smooth vorticity $\omega \in \bigwedge^2(\R^2)$
the coadjoint orbit is
\begin{align*}
  \Orb(\omega) := \{ \varphi_* \omega \mid \varphi \in \SDiff(\R^2) \}.
\end{align*}
This may or may not be a manifold.
If $\Orb(\omega)$ is a manifold, then a vector on $\Orb(\omega)$ at the point $\tilde{\omega} \in \Orb(\omega)$ is given by $\pounds_u[\tilde\omega]$ for some (non-unique) $u \in \mathfrak{X}(\R^2)$. There is a symplectic structure on $\Orb(\omega)$
given by
\begin{align*}
  \Omega_\omega( \pounds_u[\omega] , \pounds_v[\omega] ) = \int \omega( u(x), v(x) ) \mu.
\end{align*}
where $\mu$ is a volume form on $\R^2$.
This formula is derived on page 313 of \cite{MarsdenWeinstein1983}
but it is nothing more than a special case of the KKS theorem \cite[see the boxed formula on p.303]{FOM}.
We would like to generalize it to the case of non-smooth vorticities such
as point vorticies.  To do this, we can define the function $f_\omega = \frac{\omega}{\mu}$.  Then the symplectic form takes the form
\begin{align*}
  \Omega_\omega( \pounds_u[\omega] , \pounds_v[\omega] ) = \int f_{\omega}(x) \mu( u(x), v(x) ) \mu = \langle f_{\omega}, \mu(u,v) \rangle_{L^2(\mu)}.
\end{align*}
We can identify $\omega$ with the distribution $\langle f_\omega , \cdot \rangle_{L^2(\mu)}$.
That is to say
\begin{align*}
  \langle \omega , \phi \rangle := \langle f_\omega , \phi \rangle_{L^2(\mu)}
\end{align*}
for all $\phi \in C^\infty(\R^2)$.
This allows us to let $\omega$ be something like a Dirac-delta,
or a distributional dirative of a Dirac-delta.
Moreover, the symplectic form is written as
\begin{align*}
  \Omega_\omega( \pounds_u[\omega], \pounds_v[\omega]) = \langle \omega , \mu(u,v) \rangle.
\end{align*}
In the case that $\mu = dx \wedge dy$ we can 
identify $\mu(u,v)$ with the function $u^x v^y - u^y v^x = u \times v$ so that
\begin{align*}
  \Omega_\omega( \pounds_u[\omega], \pounds_v[\omega]) = \langle \omega , u^x v^y - u^y v^x \rangle.
\end{align*}


\begin{thm}
	Let $\omega_0 = \gamma_i^\alpha \partial_{\alpha} \delta_{Z_i}$ for $\gamma_i^\alpha \neq 0$ and let
	$M^{(k)} = \{ \sum_{|\alpha| \leq k} \Gamma_i^\alpha \partial_{\alpha} \delta_{Z_i} \}$.
	Then $\Orb(\omega_0)$ is a sub-manifold of the $M^{(k)}$.
\end{thm}
\begin{proof}
	Let $\varphi \in \SDiff(\R^2)$.  Then
	\begin{align*}
		\langle \varphi_* \omega_0 , f \rangle &= \langle \omega_0 , f \circ \varphi \rangle \\
                &= \gamma_i^\alpha \partial_{\alpha}|_{z=Z_i} (f \circ\varphi)(z) \\
                &= \gamma_i^\alpha \partial_{\beta}|_{\varphi(Z_i)} f \partial_\beta|_{Z_i} \varphi^\alpha.
	\end{align*}
        Where we have used the Faa di Bruno formula using the indexing convention in my own article \cite{Jacobs2014b}.
        Thus $\varphi_* \omega_0 \in M^{(k)}$.
\end{proof}

If we would like a more geometric description of what $M^{(k)}$ is,
we could identify $M^{(k)}$ as a dual vector-bundle
\begin{align*}
  M^{(k)} = \left(\Jet^{(k)}( \R^2 , \R)^n \right)^*
\end{align*}
where we view $\Jet^{(k)}( \R^2 , \R)$ as a vector-bundle over $\R^2$.
This identification is tautological upon noting that for $\omega \in M^{(k)}$
the quantity $ \langle \omega , f \rangle$
is a linear function of the $k$th order Taylor-coefficients of $f \in C^\infty(\R^2)$
about some points $z_1,\dots,z_n$.

Finally, as a sanity check, we can also prove that the equations of motion derived previously are identical to
the equations of motion obtained using this symplectic structure.

\begin{thm}
	Let $H(\omega) = \frac{1}{2} \langle \omega , K*\omega \rangle_{L^2}$.  Then Hamilton's equations on $\Orb(\omega_0)$
	are given by
	\begin{align*}
		\partial_t \omega + \pounds_u[\Omega] \quad,\quad u = \nabla^\perp \psi \quad,\quad \psi = K*\omega.
	\end{align*}
\end{thm}
\begin{proof}
  Let $\omega \in \Orb(\omega_0)$, and let $X_H(\omega) = \pounds_u[\omega]$ for some (non-unique) $u \in \mathfrak{X}_{\rm div}(\R^2)$.
  Out goal is to solve for $u$.
  By the definition of $X_H$ we see that for any
  $v \in \mathfrak{X}_{\rm div}(\R^2)$
  \begin{align*}
    \langle \omega , u^x v^y - u^y v^x \rangle &=
    \Omega_{\omega}( \pounds_u[\omega] , \pounds_v[\omega] ) = \langle dH(\omega) , \pounds_v[\omega] \rangle \\
    &= \langle K * \omega , \pounds_v[\omega] \rangle 
  \end{align*}
  If we let $\psi := K*\omega$ then
  \begin{align*}
    = - \langle \omega , \pounds_{v}[\psi] \rangle 
    = \langle \omega , v^x \partial_x \psi + v^y \partial_y \psi \rangle
  \end{align*}
  We see that $u = (-\partial_y \psi,\partial_x \psi)$
  would be a possible solution.
  As $\Omega$ is non-degenerate, this is the unique solution.
\end{proof}

\subsection{The first order case}
Let us deal with the first order case.
\begin{thm}
Let $Z_1,\dots,Z_n \in \R^2$ be distinct.
The coadjoint orbit of
\begin{align*}
  \omega = \sum_{i=1}^N \gamma_i \delta_{Z_i} + \gamma_i^x \partial_x \delta_{Z_i} + \gamma_i^y \partial_{y} \delta_{z_i}
\end{align*}
is
\begin{align*}
  \Orb(\omega) &= \left\{ \sum_{i=1}^n \gamma_i \delta_{z_i} + \Gamma_i^x \partial_x \delta_{\tilde{z}_i} + \Gamma_i^y \partial_{y} \delta_{z_i}
  \mid z_i \in \R^2, (\Gamma_i^x,\Gamma_i^y) \in \R^2 \backslash \{0\} \right\} \\
  &\cong  \{ (z_1,\dots,z_n,\Gamma_1,\dots,\Gamma_n) \mid z_i \in \R^2, \Gamma_i \in \R^2 \backslash \{0\} , ( i \neq j \implies z_i \neq z_j ) \}.
\end{align*}
The symplectic structure is
  \begin{align*}
    \Omega( (\dot{z},\dot{\Gamma}), (\delta z,\delta \Gamma) ) &=
    \gamma_i ( \dot{x}_i \cdot \delta y_i - \delta x_i \cdot \dot{y}_i ) \\
    &\quad + \dot{\Gamma}_i^x \cdot \delta y_i
    - \dot{\Gamma}_i^y \cdot \delta x_i
    + \delta \Gamma_i^y \cdot \dot{x}_i 
    - \delta \Gamma_i^x \cdot \dot{y}_i
  \end{align*}
\end{thm}
\begin{proof}
  Let $\omega$ be as above and consider some $\varphi \in \SDiff(\R^2)$.
  We find that for any function $f$
  \begin{align*}
    \langle \varphi_* \omega \mid f \rangle := \langle \omega , \varphi^* f\rangle
    &= \gamma_i f(\varphi(Z_i)) \\
    &\quad - \gamma_i^x \partial_x \varphi^x|_{z=Z_i} \partial_x f |_{z=\varphi(Z_i)}
    - \gamma_i^x \partial_x \varphi^y|_{z=Z_i} \partial_y f |_{z=\varphi(Z_i)} \\
    &\quad - \gamma_i^y \partial_y \varphi^x|_{z=Z_i} \partial_x f |_{z=\varphi(Z_i)}
    - \gamma_i^y \partial_y \varphi^y|_{z=Z_i} \partial_y f |_{z=\varphi(Z_i)}
  \end{align*}
  Collecting like terms we find
  \begin{align*}
    \varphi_* \omega = \gamma_i \delta_{\varphi(Z_i)} + \Gamma_i^x \partial_x \delta_{\varphi(Z_i)} + \Gamma_i^y \partial_y \delta_{\varphi(Z_i)}
  \end{align*}
  where
  \begin{align*}
    \Gamma = 
    \begin{bmatrix}
      \Gamma_i^x \\ \Gamma_i^y 
    \end{bmatrix}
    =
    D\varphi(Z_i) \cdot
    \begin{bmatrix}
      \gamma_i^x \\ \gamma_i^y
    \end{bmatrix}
  \end{align*}
  By varying $\varphi$ we can obtain 
  any collection of distinct points $z_1,\dots,z_n \in \R^2$
  and any collection of non-zero vectors $\Gamma_1,\dots,\Gamma_n \in \R^2 \backslash \{0\}$.
  This proves the first claim.

  To derive the symplectic structure recall the symplectic structure for a general vorticity.
  \begin{align}
    \Omega( \pounds_u[\omega] , \pounds_v[\omega] )
    = \langle \omega , u \wedge v \rangle \label{eq:vorticity_symplectic_form}
  \end{align}
  where we view $\omega$ as a distribution and $u \wedge v = u^x v^y - v^x u^y$
  as a real-valued function.
  Now let $\omega = \gamma_i \delta_{z_i} + \Gamma_i^x \partial_x\delta_{z_i} + \Gamma_i^y \partial_y \delta_{z_i}$.
  In this case the right hand side can be computed to be
  \begin{align*}
    \langle \omega , u \wedge v \rangle &= \gamma_i ( u^x(z_i)v^y(z_i) - v^x(z_i) u^y(z_i) ) \\
    &\quad + \Gamma_i^x ( u^x_{,x} v^y + u^x v^y_{,x} - v^x_{,x}u^y - v^x u^y_{,x})|_{z = z_i} \\
    &\quad + \Gamma_i^y ( u^x_{,y} v^y + u^x v^y_{,y} - v^x_{,y}u^y - v^x u^y_{,y})|_{z = z_i}
  \end{align*}
  Note that this is writen entirely in terms of the 1-jet of $u$ and $v$ evaluated 
  at $z_i$.
  Moreover, $\pounds_u[\omega] = \gamma_0 u^x(z_i) \partial_{z_i} + \dots$
  also has the property that it is only dependent on
  the one-jet of $u$ and $v$ at the poinst $z_1,\dots,z_n$.
  Therefore both side of \eqref{eq:vorticity_symplectic_form}
  can be written as a function of the finite collection of 
  numbers $u(z_i), Du(z_i), v(z_i),Dv(z_i)$.
  The result then follows by noting
  \begin{align*}
    u(z_i) \mapsto u_{z_i}\\
    u^x_{,x}(z_i) \Gamma^x_i + u^x_{,y}(z_i) \Gamma^y_i \mapsto \dot{\Gamma}_i^x \\
    u^y_{,x}(z_i) \Gamma^x_i + u^y_{,y}(z_i) \Gamma^y_i \mapsto \dot{\Gamma}_i^y.
    \end{align*}
    under the tangent lift of the coordinate chart
    \begin{align*}
      \sum_i \gamma_i \delta_{z_i} + \Gamma_i^x \partial_x \delta_{z_i} + \Gamma_i^y \partial_y \delta_{z_i} \mapsto (z_1,\dots,z_n,\Gamma_1,\dots,\Gamma_n)
      \end{align*}
\end{proof}

\section{Approximation theory}
\label{sec:approximation}
In this section we will illustrate how using jet-vortices improves the accuracy of approximation of a vorticity field in a distributional sense
with respect to an RKHS.  Let $H(\omega) = \frac{1}{2} \langle \omega , K*\omega\rangle_{L^2} =: \frac{1}{2} \langle \omega , \omega \rangle_K$, for a kernel $K:\R^2 \to \R$.
Let $h > 0$ be small and define $h \mathbb{Z}^2 = \{ (ah,bh) \in \R^2 \mid (a,b) \in \mathbb{Z}^2 \}$.
Given an $\omega \in \mathcal{D}'(\R^2)$, we can attempt to approximate $\omega$ via Dirac-deltas supported on $h \mathbb{Z}^2$.
There is a natural way to do this with respect to the Hilbert-norm because we have a reproducing kernel.
We could define $\omega_h^{(0)} = \sum_{i \in \mathbb{Z}^2} \gamma_i \delta_{z_i}$ by requiring $\omega_h^{(0)} - \omega$ be orthogonal to $\delta_z$ for each $z \in h \mathbb{Z}^2$.
This means that $K*\omega(z) =  \sum_i \gamma_i K(z-z_i)$ for each $z \in h\mathbb{Z}^2$.
Thus $\psi_h^{(0)} = \sum_i \gamma_i K(z-z_i)$ can be seen as a $0$th order approximation to $\psi = K*\omega$
because $\psi_h^{(0)}(z) = \psi(z)$ on the grid.
Assuming $\psi$ is continous, the error is $o(1)$ in $h$.
Moreover, $\omega^{(0)}_h$ serves as a $o(1)$ approximation to $\omega$ in the distributional sense.
Let $L$ be the psuedo-differential operator dual to $K*$.
Then $\langle \omega^{(0)}_h - \omega , f \rangle = \langle \psi - \psi_h^{(0)} , L[f] \rangle_{L^2}$,
and the sup-norm of $\psi-\psi_h^{(0)}$ is order $o(1)$, so that the hole expression is $o(1)$
for any $f$ in the RKHS produced by $K$.
 
The same reasoning applies if we consider $\omega^{(k)}_h = \sum_{i,\alpha} \Gamma_i^\alpha \partial_\alpha \delta_{z_i}$.
We define the scalars $\Gamma_i^\alpha$ via the equations
\begin{align*}
  \partial_\beta \psi (z_i) = \sum_j \Gamma_j^\alpha \partial_{\alpha\beta} K(z_i - z_j)
\end{align*}
for $\psi = K*\omega$, $z_i \in h \mathbb{Z}^2$, and $|\beta| \leq k$.
Then $\psi^{(k)}_h = \sum_{i,\alpha}\Gamma_k^\alpha \partial_\alpha K_{z_i}$
serves as a $o(h^{k})$ approximation of $\psi$ when $\psi \in C^k$
and $\omega^{(k)}_h$ serves as a $o(h^k)$ approximation of $\omega$ in
a distributional sense.

\section{Conservation laws}
\label{sec:conservation}
For any solution $\omega_t \in \mathcal{D}'(\R^2)$, of the EPDiff equation
\begin{align*}
  \partial_t \omega + \pounds_u [\omega] = 0 \quad,\quad u = \nabla^\perp \psi \quad,\quad \psi = K*\omega
\end{align*}
we know that $\omega_0 = \varphi_t^* \omega_t$ is constant
when $\varphi_t$ is the flow of $u$.
This conservation law then implies Kelvin's circulation theorem
using the construction of \cite{HolmMarsdenRatiu1998}
when $\omega_t$ is an exact two-form.
In this case for each loop $\gamma$
we can define the current, $\mathcal{K}(\gamma) \in  \left( d \Omega^1 (\R^2) \right)^*$
\begin{align*}
  \langle \mathcal{K}(\gamma) , \omega \rangle = \int_{\gamma} \alpha
\end{align*}
where $\omega = d\alpha$.
Then $\mathcal{K}( \varphi (\gamma) ) = \mathcal{K}(\gamma) \circ \varphi^*$, and Kelvin's theorem follows.
This construction should work just fine even if $\omega \in M^{(k)}$ (i.e. when it is distributional).

Here is a guess.
Let $L$ be the psuedo-differential operator on $C^\infty(\R^2,\R^2)$
such that $\mathrm{curl} \circ L[ \nabla^\perp K_x] = \delta_x$.
Then for $\omega \in \mathcal{D}'(\R^2)$ let $\alpha = L[\nabla^\perp (K*\omega)]$ so that $\omega = d\alpha$.  Then $\langle \mathcal{K}(\gamma) , \omega \rangle$ can be defined as before.


% When $\omega_0 \in M^{(k)}$ this construction needs to be amended because $\omega_0$ is a distribution, and not an exact two-form.
% However, something can be said imediately.
% If we choose a loop $\gamma(s)$ and use Stokes theorem
% by denoting the interior surface of $\gamma$ by $I_\gamma$
% we could build the Kelvin quantity $\mathcal{K}(\gamma)$
% in a way which mimics \cite{HolmMarsdenRatiu1998}.
% In this case the conserve quantity is
% \begin{align*}
%   \langle \mathcal{K}(\gamma) , \Gamma_i^\alpha \partial_{\alpha}\delta_{z_i} \rangle = \langle \Gamma_i^\alpha \partial_\alpha \delta_{z_i} , 1_{I_{\gamma}} \rangle
% \end{align*}
% If $\gamma$ does not intersect any of the vortices, then only the $0$th order terms pop-out.
% I'm not sure if the above construction is well defined if $\gamma$ intersects the vortices though.


\subsection{Why are there no dual pairs?}
\label{sec:dual_pairs}
We one considers a dual pair, the conservation law associated
to one of legs is only siginificant if the other leg is not injective.
This is because the conservation law is determined by the the isotropy group
of the other leg.  If this isotropy group is trivial, then there is no
conservation law.

In our case we have a momentum map of concern is the injection of $\Orb(\omega)$ into $\mathfrak{X}^*(\R^2) \equiv \mathcal{D}'(\R^2)$.
It is an injective map, and therefore the isotropy group is trivial.


% \section{Conjectures}
% \label{sec:conjecture}
% It would be nice to have a compact geometric interpretation for the quantities $\Gamma_i^\alpha$.  It seems that we can identify this collection of objects with the dual vector-bundle to $\Jet^{(k)}( S ; \R)$
% viewed as a vector-bundle over $S$.
% As a saninty check we know that $\Jet^{(0)}( S ; \R) = S \times \R$.
% The dual vector space is $\S \times \R$, and is represented
% as a standard point vortex $\gamma \delta_z$ at $z \in S$.
% The space $\Jet^{(1)}( S ; \R) = T^*S \times \R$, so that the dual space is $TS \times \R$ and is represented as a point vortex of the form $\gamma \delta_z + \Gamma \partial_{\Gamma} \delta_z$ for $\Gamma \in T_z S$.

% \subsection{Symplectic structures}
% \label{sec:conjecture_symplectic}
% In this section we attempt to link the symplectic structure of the previous
% section with that studied in the geometric mechanics literature.
% We will obtain a symplectic structure which we conjecture is equivelent to
% the symplectic structure provided in the previous section.
% This means of writing the symplectic structure
% would be less tractable from the perspective of computation,
% but more tractable from the perspective of symplectic geometry.
% Let $(V,\nu)$ be a volume manifold and $(S,\omega)$ be symplectic.
% Let $\Emb(V;S)$ denote the space of embeddings of $V$ into $S$.
% An element $v_f \in T_f \Emb(V,S)$ is a map $v_f: V \to TS$
% such that $\tau_{S} \circ v_f = f$.
% We find that $\Emb(V;S)$ is equipped with the symplectic structure
% \begin{align*}
%   \Omega( v_f, w_f) = \int_V \omega(v_f,w_f)  \nu
% \end{align*}
% The group of symplectomorphisms of $S$ acts symplectically by left-action
%  on $\Emb(V;S)$, while the group of volume-preserving diffeomorphisms
% acts symplectically on the right.  The resulting momentum maps form a dual pair \cite{MarsdenWeinstein1983,GayBalmazVizman2012}.

% In this section we seek to mimic this construction
% by replacing $\Emb(V;S)$ with the space of jets of embeddings.
% Let $\Lambda = \{ z_1 , \dots , z_n \} \subset V$
% be a collection of points in $V$ and
% consider the space $\Jet^{(k)}_\Lambda(V ; S)$ of jets of maps from
% $V$ into $S$ sourced at $\Lambda$.  We will prove that this space
% is a symplectic manifold upon choosing an element $\mu \in [ \Jet^{(k)}_\Lambda(V;\R) ]^*$.  This ``$\mu$'' produces a symplectic form on $\Jet_{\Lambda}^{(k)}$
% in the same that``$\nu$'' did so in the case of $\Emb(V;S)$.
% \begin{prop}
%   Let $\mu$ be a non-degenerate element of $[\Jet^{(k)}_\Lambda( V;\R)]^*$.
%   Then $\Jet_{\Lambda}^{(k)} ( V ; S)$ is a symplectic manifold
%   with symplectic form
%   \begin{align*}
%     \Omega( v_j , w_j ) = \langle \mu  \mid  \Jet^{(k)}_\Lambda [ \omega( v_f , w_f ) ] \rangle.
%   \end{align*}
%   where $f \in \Emb(V;S)$ is an arbitrary representative
%   such that $\Jet^{(k)}_\Lambda(f) = j$
%   and $v_f,w_f \in T_f\Emb(V;S)$ are arbitrary up to the
%   constraints $v_j = T\Jet_{\Lambda}^{(k)}( v_f)$
%   and $w_j = T\Jet_{\Lambda}^{(k)}( w_f)$.
% \end{prop}

% \begin{proof}
%   \begin{enumerate}
%     \item Verify that the expression is well defined.
%     \item Verify that it is anti-symmetric.
%     \item Verify that it is non-degenerate.
%     \item Verify that it is closed.
%   \end{enumerate}
%   \todo[inline]{I still need to do this}
% \end{proof}


\bibliographystyle{amsalpha}
\bibliography{/Users/hoj201/Dropbox/hoj_2015}
\end{document}
